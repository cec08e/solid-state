\title{Solid State Physics: Final Exam Review}
\date{\today}

\documentclass[10pt]{article}
\usepackage{amsthm}
\usepackage{subcaption}
\usepackage{listings}
\usepackage{graphicx}
\usepackage{physics}

\graphicspath{ {figures/} }
\begin{document}
\maketitle

\section{LCAO and Tight Binding}
\subsection{Gentle Introduction: Covalent Bonding of Hydrogen Atoms}
Imagine a system in which two Hydrogen atoms are held with fixed-position nuclei
("Born-Oppenheimer Approximation") and a shared electron between them.
Goal: calculate the eigenenergies of the system as a funtion of the distance from the fixed nuclei.


The Hamiltonian of the system is given by

$$
H = K + V_{1} + V_{2}
$$

where
$$
K = \frac{\textbf{p}^{2}}{2m}
$$
and the coulombic potential due to the nucleus fixed at position $\textbf{R}_{i}$ is given by
$$
V_{i} = \frac{e^{2}}{4\pi\epsilon_{0}|\textbf{r} - \textbf{R}_{i}|}
$$
We can write down a trial solution of the form
$$
\ket{\psi} = \phi_{1}\ket{1} + \phi_{2}\ket{2}
$$
where $\ket{i}$ are the atomic orbitals (or ``tight-binding orbitals'') representing the
ground state solution for that particular isolated nucleus. That explicitly means the following
$$
(K + V_{1})\ket{1} = \epsilon_{0}\ket{1}
$$
$$
(K + V_{2})\ket{2} = \epsilon_{0}\ket{2}
$$
where $\epsilon_{0}$ is the ground state energy of a single Hydrogen atom.

In the LCAO/tight-binding method, we make the following approximation: \textbf{the
atomic orbitals $\ket{i}$ are orthogonal} such that
$$
\bra{i}\ket{j} = \delta_{ij}
$$
The Schrodinger equation can be written as
$$
H\ket{\psi} = E\ket{\psi}
$$
or, alternatively, in the $\ket{i}$ basis:
$$
\begin{bmatrix}
  H_{11} & H_{12} \\
  H_{21} & H_{22}
\end{bmatrix}
\begin{bmatrix}
  \phi_{1} \\
  \phi_{2}
\end{bmatrix}
 = E \begin{bmatrix}
   \phi_{1} \\
   \phi_{2}
 \end{bmatrix}
$$
By using a variational method where
$$
E = \frac{\bra{\psi}H\ket{\psi}}{\bra{\psi}\ket{\psi}}
$$
and
$$
\frac{\partial E}{\partial \phi_{i}} = \frac{\partial E}{\partial \phi_{i}*} = 0
$$
We obtain an eigenvalue equation
$$
\sum_{j}H_{ij}\phi_{j} = E\phi_{i}
$$
where $H_{ij} = \bra{i}H\ket{j}$. The components of $H$ can be written explicitly as
$$
H_{11} = \bra{1}H\ket{1} = \bra{1}K+V_{1}\ket{1} + \bra{1}V_{2}\ket{1} = \epsilon_{0} + V_{cross}
$$
$$
H_{22} = \bra{2}H\ket{2} = \bra{2}K+V_{2}\ket{2} + \bra{2}V_{1}\ket{2} = \epsilon_{0} + V_{cross}
$$$$
H_{12} = \bra{1}H\ket{2} = \bra{1}K+V_{2}\ket{2} + \bra{1}V_{1}\ket{2} = -t
$$$$
H_{21} = \bra{2}H\ket{1} = \bra{2}K+V_{1}\ket{1} + \bra{2}V_{2}\ket{1} = -t^*
$$
We make the following observations:
\begin{itemize}
  \item The on-site energy is given by
  $$\bra{i}K + V_{i}\ket{i} = \epsilon_{0}$$
  \item The coulombic potential due to site $j$ on site $i$ is given by
  $$\bra{i}V_{j}\ket{i} = V_{cross}$$
  \item The \emph{hopping term} is defined by
  $$\bra{j}V_{j}\ket{i} = \bra{i}V_{i}\ket{j}^* = -t$$
\end{itemize}

The eigenvalue equation then takes on the form of a $2\times 2$ matrix equation
$$
\begin{bmatrix}
  \epsilon_{0} + V_{cross} & -t \\
  -t^{*} & \epsilon_{0} + V_{cross}
\end{bmatrix}
\begin{bmatrix}
  \phi_{1} \\
  \phi_{2}
\end{bmatrix}
= E\begin{bmatrix} \phi_{1} \\ \phi_{2} \end{bmatrix}
$$
Diagonalization yields the eigenenergies
$$
E_{\pm} = \epsilon_{0} + V_{cross} \pm |t|$$

\subsection{Tight-Binding Chain}
In this section, we seek to observe that all waves in periodic environments behave similarly. Here
we consider electron waves, but we should consider the similarities to vibrational waves (phonons) as
well.

The one-dimensional tight binding has the following description.
\begin{itemize}
  \item There is a single orbital on atom $n$, denoted $\ket{n}$.
  \item Periodic boundary conditions are imposed such that $\ket{N} = \ket{0}$.
  \item Atomic orbital states are orthogonal.
  $$\bra{n}\ket{m} = \delta_{nm}$$
  \item The general trial wavefunction has the form
  $$\ket{\Psi} = \sum_{n}\phi_{n}\ket{n}$$
  \item The effective Schrodinger equation is
  $$
  \sum_{m}H_{nm}\phi_{m} = E\phi_{n}
  $$
  where $H_{nm} = \bra{n}H\ket{m}$.
  \item The Hamiltonian can be written
  $$
  H = K + \sum_{j}V_{j}
  $$
  where $K = \textbf{p}^{2}/2m$ and $V_{j} = V(\textbf{r} - \textbf{R}_{j})$.
\end{itemize}
Given the description above, we have
\begin{equation}
\begin{aligned}
  H\ket{m} & = (K + V_{m})\ket{m} + \sum_{j \neq m}V_{j}\ket{m} \\
           & = \epsilon_{atomic}\ket{m} + \sum_{j \neq m}V_{j}\ket{m}
\end{aligned}
\end{equation}
Therefore,
$$
H_{nm} = \bra{n}H\ket{m} = \epsilon_{atomic}\delta_{mn} + \sum_{j \neq m}\bra{n}V_{j}\ket{m}
$$
where
$$
\sum_{j \neq m}\bra{n}V_{j}\ket{m} = \left\{\begin{matrix}
 V_{0} & n=m \\
 -t & n=m\pm 1\\
 0 & 0
\end{matrix}\right.
$$
So,
$$
H_{n,m} = (\epsilon_{atomic} + V_{0})\delta_{nm} -t(\delta_{n+1,m} + \delta_{n-1,m}) = \epsilon_{0}\delta_{nm} -t(\delta_{n+1,m} + \delta_{n-1,m})
$$

\subsubsection{Solution}
We propose an ansatz
$$\phi_{n} = \frac{e^{-ikna}}{\sqrt{N}}$$
Note the absence of of a frequency component to the exponent, which is due
to the fact that we are seeking solutions to the time-independent Schrodinger
equation.
\begin{itemize}
  \item For a system with $N$ sites, and length $L = Na$, there are $N$ possible
  solutions of the ansatz form.
  \item Each solution corresponds to $k = 2\pi m/L$, where $m = 0, ..., N-1$.
\end{itemize}
Plugging the ansatz into the Schrodinger equation gives
\begin{equation}
\sum_{m}H_{nm}\phi_{m} = \epsilon_{0}\frac{e^{-ikna}}{\sqrt{N}} - t\left( \frac{e^{-ik(n+1)a}}{\sqrt{N}} + \frac{e^{-ik(n-1)a}}{\sqrt{N}}\right)
\end{equation}
and we also know that
\begin{equation}
  E\phi_{n} = E\frac{e^{-ikna}}{\sqrt{N}}
\end{equation}
Equating the previous two equations gives us
$$
E = \epsilon_{0} - 2t\cos(ka)
$$
(Note the correspondance to the phonon spectrum for the 1D monatomic chain). The dispersion curve is shown in Fig. 1.

\begin{figure}
  \centering
    \includegraphics[width=\textwidth]{tb1}
    \caption{Dispersion curve for tight-binding chain.}
\end{figure}

Some notes about the dispersion relation:
\begin{itemize}
  \item The periodicity is $k \rightarrow k + \frac{2\pi}{a}$ (like phonons).
  \item Zero group velocity (flat curve) at the Brillouin zone boundary (like phonons).
  \item Electrons may only have eigenstates within a certain \emph{band}, referring to both
  the energy range in which the eigenstates exist as well as the individual branches
  of the dispersion curve itself.
  \item The \emph{bandwidth} refers to the difference between the maximum and minimum energies
  in a band. In Fig. 1, the bandwidth is $4t$.
  \item The bandwidth, $4t$, is dependent on the hopping amplitude and, therefore, on the
  interatomic spacing between nuclei. This dependence is shown in Fig. 2.
  \item As seen in Fig. 2, the effect of the hopping is to raise the energy of some eigenstates and lower the
  energy of others (such that the average is still $\epsilon_{0}$). If the band is not completely filled, then
  the average energy dips below $\epsilon_{0}$ since some higher states are not filled.
\end{itemize}

\begin{figure}
  \centering
    \includegraphics[width=\textwidth]{tb2}
    \caption{Bandwidth dependence on interatomic spacing.}
\end{figure}

Near the bottom of the band in Fig. 1, where $k$ is close to zero, the dispersion is approximately parabolic. For
small $k$, $cos(ka) \approx 1 - k^{2}a^{2}/2$, so

$$
E(k) \approx Constant + ta^{2}k^{2}
$$

\textbf{Note:} If the minimum were at the Brillouin zone edge, we would need to expand around $k = \pi/a$ instead.
Since the dispersion of the free electron can be written as
$$E_{free}(k) = \frac{\hbar^{2}k^{2}}{2m}$$
We can relate the two dispersion relations to calculate an \emph{effective mass} $m^{*}$ such that the dispersion at
the bottom of the band behaves as a free electron of mass $m^{*}$.
$$
\frac{\hbar^{2}k^{2}}{2m^{*}} = fa^{2}k^{2}
$$
and the effective mass is then
$$
m^{*} = \frac{\hbar^{2}}{2ta^{2}}
$$

\subsection{Band Filling}
\subsubsection{Monovalent Case}
If every atom in the one-dimensional, single-orbital tight binding model were to ``donate'' an electron,
then we would have a total of $N$ electrons. These electrons occupy the $N$ lowest-energy states in the band,
which has $N$ allowed eigenstates, but each eigenstate can be populated by a spin-up and a spin-down electron.
The band is therefore only half filled, as seen in the left of Fig. 3.

In Fig. 3, the Fermi surface (the energy level separating occupied and unoccupied states) is $\epsilon_{0}$. By providing a small bit of
energy, the Fermi sea can be shifted slightly such that the electrons take on an average net positive momentum (right side of Fig. 3) and
current is able to flow. For this reason, monovalent materials are frequently metals.

\begin{figure}
  \centering
    \includegraphics[width=\textwidth]{tb3}
    \caption{Monovalent Fermi surface and shifted Fermi sea.}
\end{figure}

\subsubsection{Divalent Case}
In the divalent case, there are a total of $2N$ atoms, which fill all spin-up and spin-down states of each of the $N$ allowed
eigenstates. That is, the band is completely filled. Therefore, there are no free $k$ states to which the Fermi sea could shift to allow a current to
be induced. \textbf{A filled band carries no current}. This results in a \emph{band insulator}.

\subsection{Multiple Bands}
In the one-dimensional, single-orbital tight-binding model, there is a single band. However, if we consider multiple
orbitals per unit cell, more bands will emerge. We may consider, for example:
\begin{itemize}
  \item Multi-atom unit cells with one or more orbitals each.
  \item Single-site unit cells with multiple orbitals per site.
\end{itemize}
The energy bandwidth as a function of inter-atomic spacing is shown in Fig. 4 for the case where there are two orbitals per single-site.
\begin{figure}
  \centering
    \includegraphics[width=\textwidth]{tb4}
    \caption{Energy bandwidth as a function of inter-atomic distance. Two atomic orbitals per site.}
\end{figure}
The dispersion relation for the two-site, single orbital case is shown in Fig. 5. Some notes:
\begin{itemize}
  \item If both species are divalent, then there are already a total of two electrons filled each orbital on every atom - that is,
  both branches of the dispersion are filled.
  \item If both species are monovalent, then only the bottom band is filled (i.e. only half the states are filled). So, there are $k$
  states available to shift the Fermi sea (around the Fermi surface, which is halfway between the top and bottom bands). However, to
  shift the sea, we need to apply an electric field strong enough to overcome the energy gap between the bands. \textbf{A filled band
  is an insulator as long as there is a finite gap to any higher-level band}.
  \item As seen in Fig. 4, as the interatomic distance decreases, the bandwidths of the individual bands come to overlap and there is no
  longer a gap between the lower- and higher-energy bands. In this limit, the material comes to behave as a metal.
\end{itemize}
\begin{figure}
  \centering
    \includegraphics[width=\textwidth]{tb5}
    \caption{Dispersion relation for tight-binding model with two sites per unit cell, one orbital per site.}
\end{figure}

\section{Nearly Free Electron Model}
The difference between the nearly free electron model and the tight binding model:
\begin{itemize}
  \item In the nearly free electron model, electrons are considered as free electron waves that are only very
  weakly perturbed by the periodic potential of the atoms in the solid.
  \item In the tight binding model, electrons are considered very tightly bound to atoms with a possible weak
  hopping to other atoms.
\end{itemize}
The nearly free electron model starts with the free electron case:
\begin{itemize}
  \item The free electron Hamiltonian is
  $$
     H_{0} = \frac{\textbf{p}^{2}}{2m}
  $$
  \item The eigenstates are the plane waves $\ket{\textbf{k}}$ with
  eigenenergies
  $$
    \epsilon_{0}(\textbf{k}) = \frac{\hbar^{2}|\textbf{k}|^{2}}{2m}
  $$
\end{itemize}
We now introduce a weak periodic potential $V(\textbf{r})$.
\begin{itemize}
  \item For any lattice vector $\textbf{R}$, $V(\textbf{r}) = V(\textbf{r} + \textbf{R})$.
  \item The matrix elements of $V(\textbf{r})$ are given by
  $$
    V_{k',k} = \bra{\textbf{k}'}V\ket{\textbf{k}} = \frac{1}{L^{3}}\int d^{3}r \,e^{i(\textbf{k} - \textbf{k}')\cdot \textbf{r}}\,V(\textbf{r})
    \equiv V_{\textbf{k}' - \textbf{k}}
  $$
  \item The integral in the previous point will vanish unless the Laue condition is met. Therefore,
  $$
    V_{k',k} =V_{\textbf{G}}\delta_{\textbf{G}, \textbf{k}' - \textbf{k}}
  $$
  In words, this means that a plane-wave state $\ket{\textbf{k}}$ can only scatter into another
  plane-wave state $\ket{\textbf{k}'}$ if $\textbf{k}$ and $\textbf{k}'$ are separated by a
  reciprocal lattice vector $\textbf{G}$ - this is conservation of crystal momentum.
\end{itemize}
We now apply perturbation theory. To second order, the shift in the eigenenergies due to the
potential $V(\textbf{r})$ is given by
\begin{equation}
  \begin{aligned}
  \epsilon(\textbf{k}) & = \epsilon_{0}(\textbf{k}) + \bra{\textbf{k}}V\ket{\textbf{k}} + \sum_{\textbf{k}' = \textbf{k} + \textbf{G}}\frac{|\bra{\textbf{k}'}V\ket{\textbf{k}}|^{2}}{\epsilon_{0}(\textbf{k}) - \epsilon_{0}(\textbf{k'})} \\
  & = \epsilon_{0}(\textbf{k}) + V_{0} + \sum_{\textbf{k}' = \textbf{k} + \textbf{G}}\frac{|\bra{\textbf{k}'}V\ket{\textbf{k}}|^{2}}{\epsilon_{0}(\textbf{k}) - \epsilon_{0}(\textbf{k'})}
\end{aligned}
\end{equation}
Some notes about this result:
\begin{itemize}
  \item We may assume that $V_{0} = 0$ for simplicity since this is just a constant shift in the energy spectrum.
  \item The second-order sum is taken over all $\textbf{k}'$ for which $\textbf{G} \neq 0$.
  \item In the degenerate situation, $\epsilon_{0}({\textbf{k}})$ and $\epsilon_{0}({\textbf{k}'})$ are approximately
  equal and the sum will diverge. The conditions for this case are
  $$
     \epsilon_{0}(\textbf{k}) = \epsilon_{0}(\textbf{k}')
  $$
  $$
  \textbf{k}' = \textbf{k} + \textbf{G}
  $$
  \item In one dimension, this is satisfied for $k' = -k = \frac{n\pi}{a}$ (i.e. on the Brillouin zone boundary).
\end{itemize}

\subsection{Degenerate Pertubation Theory}
When two plane wave states $\ket{\textbf{k}}$ and $\ket{\textbf{k}'} = \ket{\textbf{k} + \textbf{G}}$ have approximately
the same energy (i.e. they are close to the zone boundaries), then we need to diagonalize this subspace. We begin with

$$
\bra{\textbf{k}}H\ket{\textbf{k}} = \epsilon_{0}(\textbf{k})
$$
$$\bra{\textbf{k}'}H\ket{\textbf{k}'} = \epsilon_{0}(\textbf{k}') = \epsilon_{0}(\textbf{k} + \textbf{G})$$
$$ \bra{\textbf{k}}H\ket{\textbf{k}'} = V_{\textbf{k} - \textbf{k}'} = V_{\textbf{G}}^{*}$$
$$\bra{\textbf{k}'}H\ket{\textbf{k}} = V_{\textbf{k}' - \textbf{k}} = V_{\textbf{G}}$$
We now write the diagonalized state as a linear combination of these plane-wave states.
$$
\ket{\Psi} = \alpha\ket{\textbf{k}} + \beta\ket{\textbf{k}'} = \alpha\ket{\textbf{k}} + \beta\ket{\textbf{k} + \textbf{G}}
$$
We obtain the Schrodinger equation
$$
\begin{bmatrix}
  \epsilon_{0}(\textbf{k}) &  V_{\textbf{G}}^{*}\\
   V_{\textbf{G}}& \epsilon_{0}(\textbf{k} + \textbf{G})
\end{bmatrix}
\begin{bmatrix}
  \alpha \\
  \beta
\end{bmatrix}
 = E \begin{bmatrix} \alpha \\ \beta\end{bmatrix}
$$
which yields the characteristic equation
$$\left ( \epsilon_{0}(\textbf{k}) - E \right)\left (\epsilon_{0}(\textbf{k} + \textbf{G}) - E \right) - |V_{\textbf{G}}|^{2} = 0$$
Two cases arise:
\begin{itemize}
  \item \emph{Case 1}: $\textbf{k}$ exactly at the zone boundary.
    In this case, $\epsilon_{0}(\textbf{k}) = \epsilon_{0}(\textbf{k} + \textbf{G})$ and the characteristic equation
    reduces to
    $$\left ( \epsilon_{0}(\textbf{k}) - E \right)^{2} = |V_{\textbf{G}}|^{2} $$
    which yields
    $$E_{\pm} = \epsilon_{0}(\textbf{k})\pm|V_{\textbf{G}}|$$
    A gap opens at the zone boundary - where once the energy was $\epsilon_{0}(\textbf{k})$, we now see two energies split by $2|V_{\textbf{G}}|$.

    Consider $V(x) = V\cos(2\pi x/a)$.
    \begin{itemize}
      \item Brillouin zone boundaries are at $k = \pi/a$ and $k' = -k = -\pi/a$.
      \item $k' - k = G = -2\pi/a$
      \item $\epsilon_{0}(k') = \epsilon_{0}(k)$
      \item The diagonalized states corresponding to $E_{\pm}$ are
      $$\ket{\Psi_{\pm}} = \frac{1}{\sqrt{2}}\left ( \ket{k} \pm \ket{k'}\right )$$
      where
      $$\ket{k} = e^{ikx} = e^{ix\pi/a}$$
      $$\ket{k'} = e^{ik'x} = e^{-ix\pi/a}$$
      Therefore,
      $$\ket{\Psi_{+}} \propto \cos(x\pi/a)$$
      $$\ket{\Psi_{-}} \propto \sin(x\pi/a)$$
    \end{itemize}
    \item The periodic potential scatters between the two plane waves $\ket{\textbf{k}}$ and $\ket{\textbf{k} + \textbf{G}}$. If these two
    plane waves have the same energy, they mix strongly to form two states: one with higher energy (concentrated on the potential maxima) and
    one with lower energy (concentrated on the potential minima. See Fig. 6.

    \begin{figure}
      \centering
        \includegraphics[width=\textwidth]{tb6}
        \caption{Probability amplitudes of diagonalized states, plotted with potential in real-space.}
    \end{figure}
  \item \emph{Case 2}: $\textbf{k}$ near the zone boundary.
  Consider a plane wave $k$ near the zone boundary such that $k = n\pi/a + \delta$. This wavevector  can scatter to
  $k' = -n\pi/a + \delta$. Plugging these wavevectors into the free-energy solution gives unperturbed energies
  $$\epsilon_{0}(k) = \epsilon_{0}(n\pi/a + \delta)) = \frac{\hbar^{2}}{2m}\left [ (n\pi/a)^{2} +2n\pi\delta/a + \delta^{2}\right]$$
  $$\epsilon_{0}(k') = \epsilon_{0}(-n\pi/a + \delta) = \frac{\hbar^{2}}{2m}\left [(n\pi/a)^{2} -2n\pi\delta/a + \delta^{2} \right]$$
  Plugging these into the characteristic equation yields
  $$
  \left (\frac{\hbar^{2}}{2m} \left [(n\pi/a)^{2} + \delta^{2} \right ] - E \right)^{2} = \left ( \frac{\hbar^{2}}{2m} 2n\pi\delta/a\right)^{2} + |V_{G}|^{2}
  $$
  From which we obtain
  $$
  E_{\pm} = \frac{\hbar^{2}((n\pi/a)^{2} + \delta^{2})}{2m} \pm \sqrt{\left (\frac{\hbar^{2}}{2m}2n\pi\delta/a \right )^{2} + |V_{G}|^{2}}
  $$
  If $\delta$ is small, then
  $$
  E_{\pm} \approx \frac{\hbar^{2}(n\pi/a)^{2}}{2m} \pm |V_{G}| + \frac{\hbar^{2}\delta^{2}}{2m}\left [ 1 \pm \frac{\hbar^{2}(n\pi/a)^{2}}{m}\frac{1}{|V_{G}|}\right]
  $$
\end{itemize}
The result of both cases is that we conclude that in the presence of some periodic potential, a gap of size $2|V_{G}|$ opens at the zone boundary and near
the zone boundary this gap is quadratic in $\delta$. See Fig. 7.
\begin{figure}
  \centering
    \includegraphics[width=\textwidth]{tb7}
    \caption{Dispersion due to small perturbation in free-electron model.}
\end{figure}
At the Brillouin zone boundaries, the dispersion is parabolic around the extrema of the bands. We can write
$$
E_{+}(G+\delta) = C_{+} + \frac{\hbar^{2}\delta^{2}}{2m^{*}_{+}}
$$
$$
E_{-}(G+\delta) = C_{-} - \frac{\hbar^{2}\delta^{2}}{2m^{*}_{-}}
$$
which gives us the effective masses
$$
m^{*}_{\pm} = \frac{m}{\left |1 \pm \frac{\hbar^{2}(n\pi/a)^{2}}{m}\frac{1}{|V_{G}} \right |}
$$

\subsection{Higher Dimensions}
In two and three dimensions, we observe the following about the nearly free electron model:
\begin{itemize}
  \item Near the Brillouin zone boundary, a gap opens due to scattering by a reciprocal lattice vector. See Fig. 8.
  \item States of energy slightly higher than the zone boundary are pushed up, while states of energy slightly
  lower are pushed down.
  \item In the one-dimensional case, for $k$ on a zone boundary, there was exactly one $k'$ such that $k' = k + G$
  and $\epsilon_{0(k) = \epsilon_{0}(k')}$. In higher dimensions, there may be multiple $k'$ that may need to be mixed
  to find the degenerate eigenstates. For example, in 2D, $(\pm \pi/a, \pm \pi/a)$ all have the same energy and are separated
  by reciprocal lattice vectors.
\end{itemize}

\begin{figure}
  \centering
    \includegraphics[width=\textwidth]{tb8}
    \caption{Dispersion due to small perturbation in free-electron model.}
\end{figure}

\section{Bloch's Theorem}
\begin{itemize}
  \item How do we know the plane-wave approach to electrons is valid since the periodic potential may be very strong (and
  therefore, perturbation theory may not be valid)?
  \item No matter how strong the periodic potential, as long as it is periodic, the crystal momentum is conserved.
  \item  \textbf{Bloch's Theorem}: An electron in a periodic potential has eigenstates of the form
  $$\Psi_{\textbf{k}}^{\alpha}(\textbf{r}) = e^{i\textbf{k}\cdot\textbf{r}}u_{\textbf{k}}^{\alpha}(\textbf{r})$$
  where $u_{\textbf{k}}^{\alpha}$ is periodic in the unit cell and the crystal momentum $\textbf{k}$ can be chosen
  within the first Brillouin zone.
  \item There may be many states at each $\textbf{k}$ and these are indexed by $\alpha$.
  \item The function $u$ is known as the \emph{Bloch function} and $\Psi$ is sometimes called the \emph{modified plane wave}.
  \item Because $u$ is periodic in the unit cell, we can write
  $$
  u_{\textbf{k}}^{\alpha}(\textbf{r}) = \sum_{\textbf{G}} \tilde{u}_{\textbf{G},\textbf{k}}^{\alpha}(\textbf{r})e^{i\textbf{G}\cdot\textbf{r}}
  $$
  which guarantees that $u_{\textbf{k}}^{\alpha}(\textbf{r}) = u_{\textbf{k}}^{\alpha}(\textbf{r} + \textbf{R})$.
  \item The modified plane wave can then be written as
  $$\Psi_{\textbf{k}}^{\alpha}(\textbf{r}) = \sum_{\textbf{G}} \tilde{u}_{\textbf{G},\textbf{k}}^{\alpha}(\textbf{r})e^{i(\textbf{G}+\textbf{k})\cdot\textbf{r}}$$
  \item The implication: \textbf{every eigenstate can be written as a sum of plane waves with differ by reciprocal lattice vectors}.
  \item The reason for Bloch's theorem is that $\bra{\textbf{k}'}V\ket{\textbf{k}}$ is zero unless the Laue condition is fulfilled:
  $$\textbf{k}' = \textbf{k} + \textbf{G}$$
  \item The Schrodinger equation is block-diagonal in $\textbf{k}$ and only planewaves that differ by $\textbf{G}$ can be mixed together.
\end{itemize}

Conclusion: electrons in a periodic potential have eigenstates labeled by the crystal momenta. Even though the potential felt by the
electron due to the atomic sites may be very strong, electrons still behave as if they do not see the atoms at all. They almost form
plane-wave eigenstates - the only difference if that they are multiplied by a periodic Bloch function $u$ and are functions of the
crystal momentum.

\section{Insulator, Semiconductor, or Metal}
\subsection{Energy Bands in One Dimension}
\begin{itemize}
  \item The number of permissible $k$-states in a single Brillouin zone is equal to the number of unit cells ($N$) in the system.
  \item One electron per unit cell $\rightarrow$ 2 electrons fill each of the bottom $N/2$ states, so they band is half full. See Fig. 9.
  \begin{figure}
    \centering
      \includegraphics[width=\textwidth]{tb9}
      \caption{Band diagrams of monovalent 1-D monatomic chain with two orbitals per unit cell. The Fermi surface is shown as a dashed line. When
      an electric field is applied, the Fermi sea shifts, inducing a current. }
  \end{figure}
  \item If two electrons are donated per unit cell, then the bottom band is completely filled. See Fig. 10.
  \begin{figure}
    \centering
      \includegraphics[width=\textwidth]{tb10}
      \caption{Band diagrams of divalent 1-D monatomic chain with two orbitals per unit cell. The Fermi surface (chemical potential)
       is shown as a dashed line, within the band gap. On the left, a direct band gap is shown. On the right, a metal is shown - if the
       higher energy band were to be lifted, an indirect band gap would be obtained.}
  \end{figure}
  \item The highest occupied band is known as the \emph{valence band}, while the lowest unoccupied band is the \emph{conduction band}.
  \item In the case of a filled valence band, a sufficiently high electric field is required to excite electrons into the conduction band.
  Below the required field strength, no movement occurs and the material acts as a \emph{band insulator}.
  \item If the band gap is $< 4eV$, the material is known as a semiconductor since thermal excitations can push some electrons into the
  conduction band at room temperature.
  \item In the case of divalent 1-D monatomic chain, typically we obtain a band insulator, but we may obtain a metal as seen in the right of Fig. 10.
\end{itemize}

\subsection{Energy bands in Two, Three Dimensions}
\begin{figure}
  \centering
    \includegraphics[width=\textwidth]{tb11}
    \caption{Two-dimensional Fermi surface of a. free electrons b. weakly periodic potential. The Fermi sea area remains constant.}
\end{figure}

\subsubsection{Two Dimensions}
Consider a square lattice of monovalent atoms, for which the Brillouin zone is also square.
\begin{itemize}
  \item One electron per atom $\rightarrow$ half-filled band.
  \item In the absence of a potential, the free electron have a circular Fermi surface, which will be precisely half
  the area of the Brillouin zone (Fig. 11 a).
  \item As a weak periodic potential is added, gaps open at the zone boundaries $\rightarrow$ states close to the
  boundary decrease in energy. These states become preferential and the Fermi surface deforms as shown in Fig. 11 b.
\end{itemize}

Consider now a square lattice of divalent atoms, for which the Brillouin zone is also square.
\begin{itemize}
  \item The free electron fermi surface is still circular, but now the area must be equal to the area of the Brillouin zone (Fig. 12 a).
  \item As the periodic potential strength is increased, states within the first BZ are lowered in energy and states in the second BZ are raised
  in energy.
  \item In the case of a strong periodic potential, the entire lower band becomes filled and the Fermi surface deforms to Fig. 12 b. The behavior is
  that of an insulator.
  \item For intermediate strength, some states remain filled in the second band (see Fig 13.). This is equivalent to the 1D case shown in Fig 10b. The
  behavior is that of a metal.
\end{itemize}


  \begin{figure}
    \centering
      \includegraphics[width=\textwidth]{tb12}
      \caption{Two-dimensional Fermi surface of divalent square lattice a. free electron case b. strong periodic potential.}
  \end{figure}


\begin{figure}
  \centering
    \includegraphics[width=\textwidth]{tb13}
    \caption{Two-dimensional Fermi surface of divalent square lattice with intermediate periodic potential.}
\end{figure}



\subsection{Three Dimensions}
Consider a 3D material of movalent atoms.
\begin{itemize}
  \item In the case of a nearly free electron metal (e.g. Potassium), the Fermi surface is almost a perfect sphere of exactly
  half the volume of the Brillouin zone.
  \item As the periodic potential gets stronger, the sphere deforms but maintains the same volume.
  \item Copper, with a very strong periodic potential, is shown in Fig. 12.


  \begin{figure}
    \centering
      \includegraphics[width=\textwidth]{tb14}
      \caption{Three-dimensional Fermi surface of monovalent Copper, fcc with a strong periodic potential.}
  \end{figure}
\end{itemize}


\subsection{Tight Binding}
\begin{itemize}
  \item In the tight binding model, we imagine a number of orbitals on each atom (or in a unit cell) and allow them to weakly
  hop to other orbitals. This spreads the eigen-energies of the atomic orbitals out into bands.
  \item The one-, two-  dimensional Hamiltonians of the tight-binding model are given by
  \begin{equation}
    H_{nm} = \epsilon_{0}\delta_{nm} - t(\delta_{n+1,m} - \delta_{n-1,m})
  \end{equation}
  \begin{equation}
    \begin{aligned}
    H_{nm} & = \epsilon_{0}\delta_{n_{x}m_{x}}\delta_{n_{y}m_{y}} - t(\delta_{n_{x}+1, m_{x}}\delta_{n_{y}m_{y}} + \delta_{n_{x}m_{x}}\delta_{n_{y}+1, m_{y}} \\
    & - \delta_{n_{x}-1, m_{x}}\delta_{n_{y}m_{y}} - \delta_{n_{x}m_{x}}\delta_{n_{y}-1, m_{y}})
  \end{aligned}
  \end{equation}
  \item The 2D solution is
  $$
  E(k) = \epsilon_{0} - 2t\cos(k_{x}a)-2t\cos(k_{y} a)
  $$
  \item Increasing the number of orbitals increases the accuracy of the solution.
  \item If the unit cell is divalent, it is important to determine whether bands overlap. If atomic orbitals are sufficiently far
  apart in energy, small hopping between atoms cannot spread the bands enough to make them overlap.
  \item In the free electron picture, the gap is proportional to $|V_{G}|$ - the limit of strong periodic potential guarantees
  that the bands do not overlap.
\end{itemize}

\subsection{Failures of Band Representation}
\begin{itemize}
  \item We have developed a picture now where the structure and filling of bands determines their properties: insulator, conductor, or semiconductor.
  \item One important effect has been neglected: Coulomb repulsion between electrons, which can be on the order of several eV.
  \item It is justifiable to neglect this interaction, but there are cases where the non-interacting picture fails.
  \item \emph{Magnets}: In ferromagnetism, spins spontaneously align due to interaction effects. Aligning all spins can lower the Coulomb energy between
  electrons and therefore may be more favorable than filling lowest-energy $k$-states with both spin states.
  \item \emph{Mott insulators}: when the coulomb interaction is so strong, that there is a huge penalty to electrons sitting at the same atom site (even
  if they have different spins). The result is a single electron per site, and the inability for electrons to move from site to site (like a traffic jam).
\end{itemize}

\subsection{Band Structure and Optical Properties}
\subsubsection{Optical Properties of Insulators and Semiconductors}
\begin{itemize}
  \item Band insulators cannot absorb photons with energies less than the band-gap energy $\rightarrow$ low-energy photons create no excitations, they just pass through.
  For example, GaAs does not absorb wavelengths $>.9$ microns (~$1.45 eV$).
  \item If an insulator or wide-band-gap semiconductor has a band gap greater than $3.2eV$, it will appear completely transparent since it cannot absorb
  any wavelength of visible light. See Fig. 15.
  \item Semiconductors with smaller bandgaps may only absorb higher energy photons. For example, CdS has a bandgap of $~2.6 eV$ and absorbs
  violet and blue light. As a result, it looks reddish.
  \item Semiconductors with very small bandgaps look black since they absorb all visible light frequencies (e.g. GaAs has a bandgap of ~$1.45 eV$).
\end{itemize}
\begin{figure}
  \centering
    \includegraphics[width=\textwidth]{tb15}
    \caption{Energy spectrum of visible light.}
\end{figure}

\subsubsection{Direct and Indirect Transitions}
\begin{itemize}
  \item The band gap size determines the \emph{minimum} energy excitation which must be made in an insulator or semiconductor.
  \item It also matter at which values of $\textbf{k}$ these maximum (valence) and minimum (conduction) energies lie.
  \item \textbf{Direct band gap}: Maximum energy of valence band aligns with minimum energy of conduction band. That is, they have the same
  $\textbf{k}$ value. See Fig 10 a.
  \item \textbf{Indirect band gap}: The values of $\textbf{k}$ differ. See Fig 10 b, but imagine that there is a band gap between the bands. In this case,
  the valence band max is at the zone boundary and the conduction band min is at $k = 0$.
  \item Both types of gaps can be present. See Fig. 16.
  \item Indirect band gap transitions are difficult to experience with exposure to light. If a photon is absorbed, both energy and momentum are absorbed. Given
  an energy $E$ in the $eV$ range, the photon momentum $\hbar|k| = E/c$ is very small.
\end{itemize}

\begin{figure}
  \centering
    \includegraphics[width=\textwidth]{tb16}
    \caption{Direct and indirect band gaps. While the indirect transition is lower energy, it is hard for a photon to excite an
    electron across an indirect band gap because photons carry very little momentum (since the speed of light, c, is large)}
\end{figure}

\subsubsection{Optical Properties of Metals}
\begin{itemize}
  \item Since metals are very conductive, photons easily excite electrons which then re-emit light $\rightarrow$ metals look shiny.
  \item Silver looks brighter than gold or copper because, even though all are monovalent, the energy width of silver's conduction
  band is greater (i.e. $t$ is large for silver) so higher-energy electronic transitions are much more possible.
  \item Copper and gold do not absorb blue and violet well (and therefore not re-emitted frequently). For silver, all visible colors and re-emitted well.
\end{itemize}

\subsection{Important Conclusions and Summary}
\begin{itemize}
  \item A material which allows for low-energy excitations is a metal. This happens when at least one band is partially filled.
  \item Band insulators and semi-conductors have only filled bands and empty bands, separated by band gaps.
  \item A semi-conductor is a band insulator with a small band gap.
  \item The valence of a material determines how many carriers populate a band, or a collection of bands if the bands overlap.
  \item The gap between bands is determined by the strength of the periodic potential. If the periodic potential is strong enough, the
  atomic limit emerges in the tight binding model, and the bands will not overlap.
  \item The band picture fails to account for electron-electron interactions and therefore fails to address Magnetism and Mott insulators.
  \item Optical properties are dependent on the energies of electronic transitions. Photons can contribute to low-momentum transitions only, so
  optical transitions over an indirect band gap are weak.
\end{itemize}

\section{Semiconductor Physics}
\subsection{Electrons and Holes}
\begin{itemize}
  \item Suppose we start with an insulator or semiconductor and excite one electron from the valence to the conduction band (See Fig. 17).
  \item The absence of the electron in the valence band is known as a \emph{hole}.
  \item While the electrical charge of an electron is negative, the electrical charge of a hole is equal and opposite.
\end{itemize}

\begin{figure}
  \centering
    \includegraphics[width=\textwidth]{tb17}
    \caption{Electrons and holes in a semiconductor.}
\end{figure}

\subsubsection{Effective Mass of Electrons}
It is useful to describe the curvature at the bottom of the band in terms of an effective mass.
\begin{itemize}
  \item Assume that near the bottom of the conduction band (at $\textbf{k} = \textbf{k}_{min}$), the energy is given by
  $$
  E = E_{min} + \alpha|\textbf{k} - \textbf{k}_{min}|^{2} + ...
  $$
  \item The effective mass is given by
  $$
  \frac{\hbar^{2}}{m^{*}} = \frac{\partial^{2}E}{\partial k^{2}} = 2\alpha
  $$
  \item The group velocity is given by
  $$\textbf{v} = \frac{\nabla_{\textbf{k}}E}{\hbar} = \frac{\hbar(\textbf{k})}{m^{*}}$$
\end{itemize}


\section{Magnetism From Interactions: The Hubbard Model}
\begin{itemize}
  \item \textbf{Itinerant ferromagnetism}: the spin sites are capable of wandering freely - e.g. a free electron gas.
  \item For free electrons, it is always lower energy to have equal number of up/down spins $\rightarrow$ result of strong Coulomb interactions
  between electrons.
  \item The \textbf{Hubbard model} attempts to understand magnetism arising from e-e interactions.
\end{itemize}
The Hubbard model is characterized by the following:
\begin{itemize}
  \item We begin with a tight-binding model with hopping parameter $t$ in $d$ dimensions. The Hamiltonian will
  be denoted $H_{0}$.
  \item The full bandwidth of the band is $4dt$ in $d$ dimensions.
  \item We refer to the number of electrons in the band per site, $x$, as the doping. Since there are $2$ spin states,
  then $x/d$ is the fraction of $k$-states that are filled in the band.
  \item The Hubbard interaction is
  $$
  H_{interaction} = \sum_{i}U n_{i\uparrow}n_{i\downarrow}
  $$
  where $n_{i\uparrow/\downarrow}$ is the number of spin up/down electrons on site $i$ and $U > 0$ is known as the repulsive Hubbard
  interaction energy, giving an energy penalty whenever two electrons sit on the same lattice site.
  \item The full Hamiltonian is
  $$H = H_{0} + H_{interaction}$$
\end{itemize}

\subsection{Itinerant Ferromagnetism}
There is competition between the potential and kinetic energy as to whether ferromagnetic states develop.
\begin{itemize}
  \item The Fermi energy is lowered by placing electrons (one spin-up, one spin-down) in the lowest $k$-states.
  \item But the Hubbard energy is lowered by aligning all electrons - in which case the Pauli exclusion principle forces them
  to be positioned on different lattice sites.
\end{itemize}

\subsubsection{Mean Field Approach}
Goal: decide quantitatively whether spins will align or not. We begin with
$$
U_{n_{i\uparrow}n_{i\downarrow}} = \frac{U}{4}(n_{i\uparrow} + n_{i\downarrow})^{2} - \frac{U}{4}(n_{i\uparrow} - n_{i\downarrow})^{2}
$$
Now, we introduce operators
$$
U_{n_{i\uparrow}n_{i\downarrow}} \approx \frac{U}{4} \left < n_{i\uparrow} + n_{i\downarrow}\right >^{2} - \frac{U}{4}\left <n_{i\uparrow} - n_{i\downarrow}\right >^{2}
$$
where
$$
\left < n_{i\uparrow} + n_{i\downarrow}\right > = x
$$
and
$$
M = \frac{\mu_{B}}{v}\left <n_{i\uparrow} - n_{i\downarrow}\right >
$$
where $v$ is the unit cell volume. Therefore,
$$
\left < H_{interaction}\right > \approx \frac{V}{v}\frac{U}{4}(x^{2} - (Mv/\mu_{b})^{2})
$$
where $V/v$ is the number of unit cells. Important conclusions:
\begin{itemize}
  \item Increasing the magnetization $M$ decreases the expectation value of the interaction energy.
  \item To determine whether spins polarize, we need to compare this energy against the effect on the kinetic energy.
\end{itemize}

\subsubsection{Stoner Criterion}
\begin{itemize}
\item Consider a system (at $T = 0$) with the same number of spin-up and spin-down electrons.
\item Let $g(E_{F})$ be the total density of states at the Fermi surface per unit volume (for both spins together).
\item Flip a small number of spins such that the spin-up and spin-down Fermi surfaces become different:
$$ E_{F,\uparrow} = E_{F} + \delta\epsilon/2$$
$$ E_{F,\downarrow} = E_{F} - \delta\epsilon/2$$
\item The difference in the number density is then
$$
  \rho_{\uparrow} - \rho_{\downarrow} = \int_{0}^{E_{F} + \delta\epsilon/2} dE \frac{g(E)}{2}  - \int_{0}^{E_{F} - \delta\epsilon/2} dE \frac{g(E)}{2}
$$
For very small $\delta\epsilon$,
$$
\rho_{\uparrow} - \rho_{\downarrow} = \delta \epsilon \frac{g(E_{F})}{2}
$$
Therefore,
$$
  M = \mu_{B}(\rho_{\downarrow} - \rho_{\uparrow}) = -\mu_{B}\delta \epsilon \frac{g(E_{F})}{2}
$$
\item The kinetic energy can be written
\begin{equation}
\begin{aligned}
K & = \int_{0}^{E_{F} + \delta \epsilon/2} dE\; E \frac{g(E)}{2} + \int_{0}^{E_{F} - \delta \epsilon/2} dE \;E \frac{g(E)}{2} \\
& = 2 \int_{0}^{E_{F}} dE \; E\frac{g(E)}{2} + \int_{E_{F}}^{E_{F} + \delta \epsilon/2} dE\; E \frac{g(E)}{2} + \int_{E_{F}}^{E_{F} - \delta \epsilon/2} dE \;E \frac{g(E)}{2} \\
& \approx K_{M=0} + \frac{g(E_{F})}{2}\left [ \left ( \frac{(E_{F} + \delta \epsilon/2)^{2}}{2} - \frac{E_{F}^{2}}{2}\right) + \left ( \frac{(E_{F} - \delta \epsilon/2)^{2}}{2} - \frac{E_{F}^{2}}{2}\right)\right] \\
& = K_{M=0} + \frac{g(E_{F})}{2}(\delta \epsilon/2)^{2} \\
& = K_{M=0} + \frac{g(E_{F})}{2}\left (\frac{M}{\mu_{B}g(E_{F})}\right)^{2}
\end{aligned}
\end{equation}
\end{itemize}
The total energy expectation value per unit volume is then
$$
E_{tot} = E_{M=0} + \left ( \frac{M}{\mu_{B}}\right)^{2}\left [ \frac{1}{2g(E_{F})} - \frac{vU}{4}\right]
$$
where $$E_{M=0} = K_{M=0} + \frac{V}{v}\frac{U}{4}x^{2}$$. From this we obtain the \textbf{Stoner Criterion}: for $U > 2/(g(E_{F})v)$, the
system energy is lowered by increasing the magnetization from zero.

\subsection{Mott Antiferromagnetism}
\begin{itemize}
\item Consider a system with doping $x = N$ - there is one electron per site.
\item When $U = 0$, we have a conductor. But when $U$ is large, we have an insulator.
\item In the absence of an external field, how do the spins align? Ferro- or antiferro-magnetically?
The answer is antiferromagnetically.
\item Consider a system with one electron per site, denoted $\ket{GS_{0}}$, which has energy $E(\ket{GS_{0}})$ in
the absence of hopping.
\item Now we add a weak hopping perturbation:
$$
 E = E(\ket{GS_{0}}) + \sum_{X}\frac{|\bra{X}H_{hop}\ket{GS_{0}}|^{2}}{E_{GS_0} - E_{X}} = E(\ket{GS_{0}}) - \frac{Nz|t|^{2}}{U}
$$
where $\ket{X}$ represents any state that can be reached by a single hop (there are $Nz$ such terms).
\item If the states were aligned ferromagnetically, no such states $\ket{X}$ could exist due to Pauli exclusion. Therefore,
antiferromagnetic configurations are preferred when hopping is allowed is the large  $U$ limit of the Mott insulating phase.
\item Allowing an electron wavefunction to spread out always lowers its energy.
\end{itemize}

\subsection{Summary}
\begin{itemize}
  \item Hubbard model includes tight-binding hopping $t$ and on-site "Hubbard" interaction U.
  \item For a partially filled band, the repulsive interaction, when strong enough, makes the system
  an itinerant ferromagnet. Aligned spins can have lower energy because they do not double occupy sites.
  This lowers the energy with respect to U, although it increases the kinetic energy.
  \item For a half-filled band, the repulsive interaction makes the Mott insulator anti-ferromagnetic: virtual
  hopping lowers the energy of anti-aligned neighboring spins.
\end{itemize}


To do: Chapters 17
\end{document}
