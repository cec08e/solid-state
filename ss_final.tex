\title{Solid State Physics: Final Exam Review}
\date{\today}

\documentclass[10pt]{article}
\usepackage{amsthm}
\usepackage{subcaption}
\usepackage{listings}
\usepackage{graphicx}
\usepackage{physics}

\graphicspath{ {figures/} }
\begin{document}
\maketitle

\section{LCAO and Tight Binding}
\subsection{Gentle Introduction: Covalent Bonding of Hydrogen Atoms}
Imagine a system in which two Hydrogen atoms are held with fixed-position nuclei
("Born-Oppenheimer Approximation") and a shared electron between them.
Goal: calculate the eigenenergies of the system as a funtion of the distance from the fixed nuclei.


The Hamiltonian of the system is given by

$$
H = K + V_{1} + V_{2}
$$

where
$$
K = \frac{\textbf{p}^{2}}{2m}
$$
and the coulombic potential due to the nucleus fixed at position $\textbf{R}_{i}$ is given by
$$
V_{i} = \frac{e^{2}}{4\pi\epsilon_{0}|\textbf{r} - \textbf{R}_{i}|}
$$
We can write down a trial solution of the form
$$
\ket{\psi} = \phi_{1}\ket{1} + \phi_{2}\ket{2}
$$
where $\ket{i}$ are the atomic orbitals (or ``tight-binding orbitals'') representing the
ground state solution for that particular isolated nucleus. That explicitly means the following
$$
(K + V_{1})\ket{1} = \epsilon_{0}\ket{1}
$$
$$
(K + V_{2})\ket{2} = \epsilon_{0}\ket{2}
$$
where $\epsilon_{0}$ is the ground state energy of a single Hydrogen atom.

In the LCAO/tight-binding method, we make the following approximation: \textbf{the
atomic orbitals $\ket{i}$ are orthogonal} such that
$$
\bra{i}\ket{j} = \delta_{ij}
$$
The Schrodinger equation can be written as
$$
H\ket{\psi} = E\ket{\psi}
$$
or, alternatively, in the $\ket{i}$ basis:
$$
\begin{bmatrix}
  H_{11} & H_{12} \\
  H_{21} & H_{22}
\end{bmatrix}
\begin{bmatrix}
  \phi_{1} \\
  \phi_{2}
\end{bmatrix}
 = E \begin{bmatrix}
   \phi_{1} \\
   \phi_{2}
 \end{bmatrix}
$$
By using a variational method where
$$
E = \frac{\bra{\psi}H\ket{\psi}}{\bra{\psi}\ket{\psi}}
$$
and
$$
\frac{\partial E}{\partial \phi_{i}} = \frac{\partial E}{\partial \phi_{i}*} = 0
$$
We obtain an eigenvalue equation
$$
\sum_{j}H_{ij}\phi_{j} = E\phi_{i}
$$
where $H_{ij} = \bra{i}H\ket{j}$. The components of $H$ can be written explicitly as
$$
H_{11} = \bra{1}H\ket{1} = \bra{1}K+V_{1}\ket{1} + \bra{1}V_{2}\ket{1} = \epsilon_{0} + V_{cross}
$$
$$
H_{22} = \bra{2}H\ket{2} = \bra{2}K+V_{2}\ket{2} + \bra{2}V_{1}\ket{2} = \epsilon_{0} + V_{cross}
$$$$
H_{12} = \bra{1}H\ket{2} = \bra{1}K+V_{2}\ket{2} + \bra{1}V_{1}\ket{2} = -t
$$$$
H_{21} = \bra{2}H\ket{1} = \bra{2}K+V_{1}\ket{1} + \bra{2}V_{2}\ket{1} = -t^*
$$
We make the following observations:
\begin{itemize}
  \item The on-site energy is given by
  $$\bra{i}K + V_{i}\ket{i} = \epsilon_{0}$$
  \item The coulombic potential due to site $j$ on site $i$ is given by
  $$\bra{i}V_{j}\ket{i} = V_{cross}$$
  \item The \emph{hopping term} is defined by
  $$\bra{j}V_{j}\ket{i} = \bra{i}V_{i}\ket{j}^* = -t$$
\end{itemize}

The eigenvalue equation then takes on the form of a $2\times 2$ matrix equation
$$
\begin{bmatrix}
  \epsilon_{0} + V_{cross} & -t \\
  -t^{*} & \epsilon_{0} + V_{cross}
\end{bmatrix}
\begin{bmatrix}
  \phi_{1} \\
  \phi_{2}
\end{bmatrix}
= E\begin{bmatrix} \phi_{1} \\ \phi_{2} \end{bmatrix}
$$
Diagonalization yields the eigenenergies
$$
E_{\pm} = \epsilon_{0} + V_{cross} \pm |t|$$

\subsection{Tight-Binding Chain}
In this section, we seek to observe that all waves in periodic environments behave similarly. Here
we consider electron waves, but we should consider the similarities to vibrational waves (phonons) as
well.

The one-dimensional tight binding has the following description.
\begin{itemize}
  \item There is a single orbital on atom $n$, denoted $\ket{n}$.
  \item Periodic boundary conditions are imposed such that $\ket{N} = \ket{0}$.
  \item Atomic orbital states are orthogonal.
  $$\bra{n}\ket{m} = \delta_{nm}$$
  \item The general trial wavefunction has the form
  $$\ket{\Psi} = \sum_{n}\phi_{n}\ket{n}$$
  \item The effective Schrodinger equation is
  $$
  \sum_{m}H_{nm}\phi_{m} = E\phi_{n}
  $$
  where $H_{nm} = \bra{n}H\ket{m}$
\end{itemize}
\end{document}
