\title{Solid State Physics: Midterm Review}

\documentclass[10pt]{article}
\usepackage{amsthm}
\usepackage{graphicx}
\usepackage{subfig}
\usepackage{physics}
\graphicspath{ {figures/} }
\begin{document}
\maketitle

\section{Drude Theory}
\subsection{Basic Model}
\begin{itemize}
  \item Theory of metallic conduction proposed by Paul Drude.
  \item Considers metal a "gas" of electrons on which the kinetic theory of gases may be applied.
  \item Assumption of kinetic theory of gases: molecules, or electrons, are identical constituents that move
  in straight lines until they encounter a collision.
  \item Assumption of metallic "gases": when atoms of a metallic element form a metal, the valence (or "conduction")
  electrons move freely while the positively charged ions remain stationary.
  \item Reality: metals have a much higher density than gas and include very strong electron-electron and electron-ion
  interactions.

\end{itemize}

The \emph{charge carrier density} $n$ may be written as
$$
n = \frac{N}{V} = N_{A}Z\frac{\rho}{A}
$$
where $N$ is the total number of conduction electrons, $V$ is the sample volume, $N_{A}$ is Avogradro's number,
$Z$ is the number of valence electrons per atom, $\rho$ is the mass density of the metal, and $A$ is the atomic mass (mass per mole of
the metal). A typical charge carrier density is on the order of $n ~ 10^{22} cm^{-3}$ for metals.

Another useful way to characterize the charge carrier density is by specifying $r_{s}$, the radius of the spherical volume containing
one charge carrier.
$$
\frac{V}{N} = \frac{1}{n} = \frac{4}{3}\pi r_{s}^{3} \rightarrow r_{s} = \left (\frac{3}{4\pi n}\right )^{1/3}
$$

The basic assumption of the Drude theory are:
\begin{itemize}
  \item \emph{Indepedent electron approximation}: electrons do not interact with one another (between collisions).
  \item \emph{Free electron approximation}: electrons do not interact with positively charged ions (between collisions).
  \item Collisions are instantaneous events that abruptly alter the electron velocity. The mechanism of scattering is not
  well-defined. Drude assumed it was only due to electron-ion scattering, but the reality is much more complicated.
  \item Electrons experience a collision with probability $1/\tau$ per unit time. The probability of a collision in time $dt$
  is $dt/\tau$. $\tau$ is the \emph{mean free time}, or \emph{relaxation time}.
  \item Electrons achieve thermal equilibrium only through collision. After each collision, an electron emerges with a randomly oriented
  velocity proportional to the local temperature $T$.
\end{itemize}

\subsection{DC Conductivity}
Ohm's Law:
$$V = IR$$
where $V$ is the voltage drop, $I$ is the current, and $R$ is the resistance. Using $\rho = RA/L$, where $L$ is the
distance over which the voltage drop occurs, and $A$ is the cross-sectional area of the sample,
$$\textbf{E} = \rho \textbf{j} $$
where $\textbf{j}$ is the current density - the amount of charge flowing through a unit area per unit time.
For $n$ electrons moving with velocity $\textbf{v}$,
$$
\textbf{j} = nq\textbf{v} = n(-e)\textbf{v}
$$
\begin{itemize}
  \item When $\textbf{E} = 0$, there is no preferred direction so $\langle \textbf{v} \rangle = 0$ and therefore $\textbf{j} = 0$.
  \item When $\textbf{E} \neq 0$, the velocity of an electron is given by
  $$ \textbf{v} = \textbf{v}_{0} + \frac{(-e)\textbf{E}t}{m}$$
  So,
  $$ \langle \textbf{v} \rangle = \langle \textbf{v}_{0} \rangle + \frac{(-e)\textbf{E}\tau}{m} = \frac{(-e)\textbf{E}\tau}{m} $$
  Where the velocity acquired over time $t$ due to a constant acceleration of $\textbf{a} = \frac{-e\textbf{E}}{m}$ is given by $\textbf{at}$.
\end{itemize}
We can therefore generally write the current density as
$$\textbf{j} = \frac{ne^{2}\tau}{m} \textbf{E}= \sigma \textbf{E} $$
where the \emph{conductivity} $\sigma$ is given by $ne^{2}\tau/m$.
If we consider the \emph{mean free path}, $l = \langle v\rangle \tau$, Drude's predictions for $\langle v\rangle$ based on kinetic theory
of gases estimate $l \approx 5 $angstroms. This is consistent with Drude's view that electron-ion collisions dominated. In reality, the mean free
path is at least an order of magnitude larger. At low temperatures, the mean free path is underestimated by a factor of $~10^{3}$.

\subsection{Drude Equation of Motion}
The time-dependent current density may be written in terms of the momentum $\textbf{p}(t)$ as
$$\textbf{j}(t) = n(-e)\textbf{v}(t) = -\frac{ne}{m}\textbf{p}(t)$$
At an infinitesimal time $dt$ later, the momentum will be given by
$$\textbf{p}(t + dt) = \left ( 1 - \frac{dt}{\tau}\right ) \left [ \textbf{p}(t) + \textbf{F}dt\right ] + \left ( \frac{dt}{\tau}\right )\mathcal{O}(\textbf{F}dt)$$
The first term comes from the fact that, with probability $1-dt/\tau$, the electron will not collide during time $dt$ and will therefore acquire the impulse $\textbf{F}dt$
in addition to its original momentum, $\textbf{p}(t)$. The second term comes from the case that the electron does collide, which happens with probability $dt/\tau$, in
which case its contribution to the momentum is $\mathcal{O}(\textbf{F}dt)$ since its velocity after collision averages over all directions to zero.
Therefore,
$$\textbf{p}(t + dt) = \textbf{p}(t) - \frac{\textbf{p}(t)dt}{\tau} + \textbf{F}dt + \mathcal{O}(dt^{2})$$
Ignoring the higher-order terms in $dt$, we obtain:
$$\frac{\textbf{p}(t + dt) - \textbf{p}(t)}{dt} = \frac{d\textbf{p}(t)}{dt} =  - \frac{\textbf{p}(t)}{\tau} + \textbf{F}$$

\subsection{Hall Effect and Magnetoresistance}
E.H. Hall predicted that the force experienced by a current due to a magnetic field would manifest as an observable transverse voltage, today known
as the Hall voltage.
The experimental setup is an applied electric field $E_{x}$ driving current in the $+\hat{x}$ direction along a sample. An applied magnetic field
is also present, $\textbf{B} = B\hat{z}$ perpendicular to the sample. Charge carriers experience a Lorentz force given by
$$qE_{x}\hat{x} + q(\textbf{v}\times\textbf{B})$$
In the case of electrons, $q = -e$, and the second term gives rise to a deflection in the $-\hat{y}$ direction. As electrons build up on the sample
surface along the $-\hat{y}$ edge, a transverse electric field $E_{y}$ is created in the $-\hat{y}$ direction. This field will grow in magnitude until is opposes
the magnetic force term. Therefore,
$$qE_{y}\hat{y} + q(\textbf{v}\times\textbf{B}) = 0$$
In the case of electrons,
$$-eE_{y} + -ev_{x}B = 0$$
Therefore, the transverse electric field has magnitude $E_{y} = -v_{x}B$.

There are two quantities of interest:
\begin{itemize}
  \item \emph{Magnetoresistance} $\rho(B) = \frac{E_{x}}{j_{x}}$. This quantity is field independent.
  \item \emph{Hall coefficient} $R_{H} = \frac{E_{y}}{j_{x}B}$. This value is negative for negative charge carriers
  and positive for positive charge carriers. Measurement of the Hall coefficient, or Hall field $E_{y}$, gives information
  about the charge carriers.
\end{itemize}

Starting with the Drude equation of motion, we can derive equations for the magnetoresistance and Hall coefficient.
$$\frac{d\textbf{p}(t)}{dt} =  - \frac{\textbf{p}(t)}{\tau} + \textbf{F} = \frac{\textbf{p}(t)}{\tau} + \left (-e\textbf{E} + \frac{-e}{m}\right)(\textbf{p}\times\textbf{B})$$
In the steady state, $\frac{d\textbf{p}}{dt} = 0$, so we obtain
$$
0 = -\frac{p_{x}}{\tau} + (-eE_{x} + \frac{-e}{m}p_{y}B) = -eE_{x} - \omega p_{y} - \frac{p_{x}}{\tau}
$$
$$
0 = -\frac{p_{y}}{\tau} + (-eE_{y} - \frac{-e}{m}p_{x}B) = -eE_{y} + \omega p_{x} - \frac{p_{y}}{\tau}
$$
where $\omega = eB/m$, the cyclotron frequency. The quantity $\omega \tau$ is a good
measure of the strength of the magnetic field. For example, if $\omega \tau$ is very small, the current
is very nearly parallel to $E_{x}$.  From these we obtain
$$
\sigma_{0}E_{x} = \omega j_{y}\tau + j_{x}
$$
$$
\sigma_{0}E_{y} = - \omega j_{x} \tau + j_{y}
$$
The Hall voltage is obtained in the limit when $j_{y} = 0$. Therefore, the steady state transverse field is given by
$$E_{y} = \frac{- \omega j_{x}\tau}{\sigma_{0}} = -\frac{B}{ne}j_{x}$$
So the Hall coefficient is given by
$$R_{H} = \frac{E_{y}}{j_{x}B} = -\frac{1}{ne}$$
This result predicts that the Hall coefficient is dependent only on the number of charge carriers. However, the actual value is
temperature and field-dependent. But the limiting value, in the case of very low temperature and very high fields, is the Drude
result.

\subsection{AC Electrical Conductivity}
We wish to calculate the current induced by a time-dependent electric field. The field can be written in the form
$$
\textbf{E}(t) = Re(\textbf{E}(\omega)e^{-i\omega t})
$$
In the absence of a magnetic field, the equation of motion becomes
$$
\frac{d\textbf{p}(t)}{dt} =  - \frac{\textbf{p}(t)}{\tau} + \textbf{F} = \frac{\textbf{p}(t)}{\tau} + -e\textbf{E}
$$
And the solution we seek is therefore of the form
$$
\textbf{p}(t) = Re(\textbf{p}(\omega)e^{-i\omega t})
$$
The equation of motion gives us
$$
-i\omega \textbf{p}(\omega) = -\frac{\textbf{p}(\omega)}{\tau} - e\textbf{E}(\omega)
$$
And we can use the fact that $\textbf{j} = -ne\textbf{p}/m$ to write
$$
\textbf{j}(t) = Re(\textbf{j}(\omega)e^{-i\omega t})
$$
and then rewrite the equation of motion as
$$
i\frac{m\omega}{ne}\textbf{j}(\omega) = \frac{m\textbf{j}(\omega)}{ne\tau} - e\textbf{E}(\omega)
$$
$$
\left[ i\frac{m\omega}{ne} - \frac{m}{ne\tau} \right ]\textbf{j}(\omega) = - e\textbf{E}(\omega)
$$
$$
\textbf{j}(\omega) = \frac{-ne^{2}}{m(i\omega - \frac{1}{\tau})}\textbf{E}(\omega)
$$
We finally obtain the expression for the frequency-dependent AC conductivity $\sigma(\omega)$:
$$
\textbf{j}(\omega) = \sigma(\omega)\textbf{E}(\omega)
$$
where
$$
\sigma(\omega) = \frac{\sigma_{0}}{1-i\omega\tau}
$$
In the case of $\omega = 0$, this reduces to the DC conductivity.
There are two complications addressed in Ashcroft/Mermin:
\begin{itemize}
  \item Neglecting the $\textbf{B}$ counterpart: this is acceptable because the term is technically scaled
  by a factor of $v/c$. In most cases, it is roughly $10^{-10}$ of the $\textbf{E}$-field term and may safely be ignored.
  \item We have assumed spatial uniformity, but the $\textbf{E}$-field actually varies in time and space. However, as long as
  the field does not vary much over a few mean free path lengths, we may safely assume spatial invariance.
\end{itemize}

\subsection{Thermal Conductivity}
The Wiedemann-Franz law states that the ratio of the thermal to the electrical conductance, $\kappa / \sigma$, is
directly proportional to the temperature.

The Drude model assumes the bulk of thermal conductance is performed by conduction electrons. To define and describe thermal conductance, consider a metal bar
along which a temperature gradient $\nabla T$ exists in the $\hat{x}$ direction. By supplying a heat source on the hot end, one can produce a uniform flow of thermal
energy that flows opposite to the sense of the thermal gradient. Let the \emph{thermal current density} $\textbf{j}^{q}$ be a vector parallel to the direction of heat flow,
whose magnitude gives the thermal energy per unit time crossing a unit cross-sectional area. Fourier's Law gives us

$$\textbf{j}^{q} = -\kappa \nabla T$$

where the proportionality constant $\kappa$ is the thermal conductivity and is positive since $\textbf{j}^{q}$ flows opposite $\nabla T$.

\subsection{Drude Theory Failures and Successes}
\begin{itemize}
  \item Drude theory fails to treat electrons using Fermi statistics.
  \item Overestimates the heat capacity.
  \item Underestimates the velocity of the electrons.
  \item Drude is successful in describing transport properties like conductivity and the Hall coefficient because
  these do not rely on velocity or specific heat.
  \item The Drude equation of motion can be thought of as the equation of motion for the Fermi sea. In the absence of a field, the
  average vector velocity of the Fermi sea is zero. But a net force shift induces a non-zero average vector velocity, known as the
  \emph{drift velocity} - this shifts the Fermi sea from the origin. The scattering process attempts to shift the Fermi sea back to the
  origin and lower the energy of the Fermi sea. The shifting happens around the Fermi surface only - electrons that are in states higher
  than the original Fermi surface will scatter to the other side of the Fermi sphere in order to lower the energy.
\end{itemize}

\section{Sommerfeld Theory}
Sommerfeld combined Fermi-Dirac statistics with the Drude model.
\subsection{Basic Fermi-Dirac Statistics}
Given a system of free electrons with chemical potential $\mu$, the probability of an eigenstate of energy $E$ bring occupied is given by the
Fermi factor
$$n_{F}(\beta(E-\mu)) = \frac{1}{e^{\beta(E - \mu)}+1}$$
At very low temperature, $n_{F} \approx 1$ for $E < \mu$ and $n_{F} \approx 0$ for $E > \mu$. So, $n_{F}$ is a step function
at low $T$. At higher $T$, it becomes more spread out.

For electrons in a box of size $V = L^{3}$, the wavefunctions of the free electron particles are given by
$$\Psi = e^{i\textbf{k}\cdot\textbf{r}}$$
Imposing PBC, we find that $\textbf{k} = (2\pi/L)(n_{1},n_{2},n_{3})$ where $n_{i}$ are integers. The corresponding
energies of the plane waves are
$$\epsilon(\textbf{k}) = \frac{\hbar |\textbf{k}|^{2}}{2m}$$
The total number of electrons in the system, $N$, is given by the sum of all possible states times the probability of each state being occupied.
Therefore,

$$N = 2 \sum_{\textbf{k}}n_{F}(\beta(\epsilon(\textbf{k}) - \mu))$$

where the factor of 2 comes from each state allowing two possible spin configurations. In the continuous limit, we can make
use of the following:

$$\int f(x)dx = \sum f(x)\delta x$$

Therefore, $N$ can be written

$$
N = 2 \frac{L^{3}}{(2\pi)^{3}}\int d\textbf{k} n_{F}(\beta(\epsilon(\textbf{k}) - \mu))
$$

Some definitions:
\begin{itemize}
  \item The \emph{Fermi energy}, $E_{F}$, is the chemical potential at $T = 0$. Sometimes also called the \emph{Fermi Level}.
  In the case of a continuum of states, the Fermi energy is the energy of the most energetic occupied electron state. In the case of a discrete
  set of states, the Fermi energy is the energy halfway between the highest occupied and lowest unoccupied states.
  \item The \emph{Fermi sea} is the set of states filled at $T = 0$.
  \item The \emph{Fermi temperature}, $T_{F}$, is given by $T_{F} = E_{F}/k_{B}$.
  \item The \emph{Fermi wavevector}, $k_{F}$, satisfies
  $$E_{F} = \frac{\hbar^{2}k_{F}^{2}}{2m}$$
  \item The \emph{Fermi momentum} $p_{F} = \hbar k_{F}$.
  \item The \emph{Fermi velocity} $v_{F} = p_{F}/m = \hbar k_{F}/m$.
\end{itemize}

\subsection{Fermi Energy at T = 0}
Consider a 3-dimensional metal with $N$ electrons at $T = 0$. The total number of electrons can be written as

$$
N = 2 \frac{L^{3}}{(2\pi)^{3}}\int d\textbf{k} n_{F}(\beta(\epsilon(\textbf{k}) - \mu))
$$

At $T = 0$, the Fermi function $n_{F}$ becomes a step function that switches at $E_{F}$ such that

$$n_{F} = \Theta(E_{F} - \epsilon(\textbf{k}))$$

Therefore,

$$
N = 2 \frac{L^{3}}{(2\pi)^{3}}\int d\textbf{k} \Theta(E_{F} - \epsilon(\textbf{k})) = 2 \frac{L^{3}}{(2\pi)^{3}} \int^{|k|<k_{F}}d\textbf{k}
$$

The final integral simply yields the volume of a ball of radius $k_{F}$. Therefore,

$$N = 2 \frac{L^{3}}{(2\pi)^{3}} \left (\frac{4}{3}\pi k_{F}^3\right )$$

The surface of this ball is known as the \emph{Fermi surface} - it is the surface dividing filled from unfilled states at $T = 0$.

This result gives us equations for the Fermi wavevector and Fermi energy:
$$
k_{F} = \left ( \frac{3}{4\pi}\frac{(2\pi)^{3} N}{2L^{3}}\right )^{1/3} = (3\pi^{2}n)^{1/3}
$$
$$
E_{F} = \frac{\hbar^{2}k_{F}^{2}}{2m} = \frac{\hbar^{2}(3\pi^{2}n)^{2/3}}{2m}
$$
For a typical metal, we can estimate the number of free electrons, and therefore calculate the Fermi energy. The Fermi energies and corresponding
Fermi temperatures are extremely large ($~7$ eV, $80000$K for copper). This indicates that only electrons that are very close to the Fermi surface
have any hope of receiving enough energy to exceed the Fermi energy. All other electrons would require an insane amount of energy to move, as no unoccupied states are nearby.


\subsection{Heat Capacity}
The heat capacity $C$ is given generally by
$$C = \frac{\partial E}{\partial T}$$

To calculate the heat capacity, we need an equation for the total energy of the system. The expectation value
of the total energy of the system is given by

$$
E_{total} = 2\frac{L^{3}}{(2\pi)^{3}}\int d\textbf{k} \epsilon(\textbf{k})n_{F}(\beta(\epsilon(\textbf{k}) - \mu))
$$
Integrating out the $theta$ and $phi$ contributions, we obtain the one-dimensional integral
$$
E_{total} = 2\frac{L^{3}}{(2\pi)^{3}}\int 4\pi k^{2} dk \epsilon(\textbf{k})n_{F}(\beta(\epsilon(\textbf{k}) - \mu))
$$
Using the relationships
$$
k = \sqrt{\frac{2\epsilon m}{\hbar^{2}}}
$$
$$
dk = \sqrt{\frac{m}{2\epsilon \hbar^{2}}}d\epsilon
$$
The total energy may be written as
$$
E_{total} = (L^{3}) \int_{0}^{\infty} \epsilon  g(\epsilon)  n_{F}(\beta(\epsilon-\mu))d\epsilon
$$
where $g(\epsilon)d\epsilon$ is the \emph{density of states per unit volume} - the total number of eigenstates with energies between $\epsilon$
and $\epsilon + d\epsilon$.
$$g(\epsilon)d\epsilon = \frac{2}{(2\pi)^{3}}4\pi k^{2}dk = \frac{(2m)^{3/2}}{2\pi^{2}\hbar^{3}}\epsilon^{1/2}d\epsilon$$
Using the expression for the Fermi energy for a 3D metal at $T=0$, we can further write
$$
g(\epsilon) = \frac{3n}{2E_{F}} \left (\frac{\epsilon}{E_{F}} \right)^{1/2}
$$

Once the chemical potential is fixed by the number of electrons in the system, the total energy may be calculated
as above. The partial derivative with respect to temperature will yield the heat capacity.

When $T = 0$, the Fermi function is a step function and the chemical potential is simply $\mu = E_{F}$.
For non-zero, but low, temperature the chemical potential is given by
$$\mu = E_{F} + \mathcal{O}(T/T_{F})^{2}$$

Let $E[T=0]$ be the kinetic energy of the system at $T=0$. For a finite temperature, the spread of the step function is roughly $\approx k_{B}T$.
That is, only electrons within a range of $k_{B}T$ of $\mu$ can be excited and they are excited within a range $k_{B}T$ above $\mu$.
The energy is then given by
$$E[T] \approx E[T=0] + \frac{\alpha}{2}[Vg(E_{F})k_{B}T](k_{B}T) = E[T=0] + \frac{\alpha}{2}D(E_{F})(k_{B}T)^{2}$$
Therefore,
$$
C = \frac{\partial E}{\partial T} = \alpha D(E_{F})k_{B}^{2}T
$$
Using
$$Vg(\epsilon) = D(\epsilon) = \frac{3N}{2E_{F}}\left ( \frac{\epsilon}{E_{F}}\right )^{1/2}$$
we obtain for the heat capacity

$$
C = \alpha \frac{3NT}{2T_{F}}
$$

\subsection{Density of States}

\subsubsection{Method 1}
The density of states $D(\epsilon)$ can be calculated from the following expression for the total energy
$$
E_{total} = \int \epsilon D(\epsilon) n_{F}(\beta(\epsilon - \mu)) d\epsilon
$$
In one dimension, the expectation value of the total energy is
$$E_{total} = 2\frac{L}{2\pi}\int d\textbf{k} \epsilon(\textbf{k}) n_{F}(\beta(\epsilon - \mu))  $$
We now make use of

$$
k = \sqrt{\frac{2m\epsilon}{\hbar^{2}}}
$$
$$
\frac{dk}{d\epsilon} = \frac{1}{2}\left ( \frac{2m\epsilon}{\hbar^{2}}\right )^{-1/2} \frac{2m}{\hbar^{2}} = \sqrt{\frac{m}{2\hbar^{2}\epsilon}}
$$
to obtain the following integral over $d\epsilon$

$$E_{total} = 2\frac{L}{2\pi}\int \sqrt{\frac{m}{2\hbar^{2}\epsilon}}\epsilon(\textbf{k}) n_{F}(\beta(\epsilon - \mu))  d\epsilon $$

Therefore,

$$D(\epsilon) = \frac{L}{\pi} \sqrt{\frac{m}{2\hbar^{2}\epsilon}}$$

Likewise, for two dimensions:

$$D(\epsilon) = \frac{L^{2}m}{\pi \hbar^{2}}$$

and for three dimensions:

$$D(\epsilon) = \frac{(2m)^{3/2}L^{3}}{2\pi^{2}\hbar^{3}}\epsilon^{1/2}$$

\subsubsection{Method 2}
To derive the same result for $D(\epsilon)$, we may first consider the total number of states $\mathcal{N}$ where, for $3$ dimensions,
$$\mathcal{N} = 2 \left (\frac{1}{\delta k}\right)^{3}\left (\frac{4\pi k^{3}}{3}\right ) = 2 \left ( \frac{L}{2\pi} \right)^{3} \left (\frac{4\pi (2m\epsilon)^{3/2}}{3\hbar^{3}} \right ) = \frac{L^{3}}{3\pi^{2}}\frac{(2m\epsilon)^{3/2}}{\hbar^{3}}$$
The density of states $D(\epsilon)$ is then given by

$$
D(\epsilon) = \frac{\partial \mathcal{N}}{\partial \epsilon} = \frac{L^{3}}{2\pi^{2}}\frac{(2m)^{3/2}(\epsilon)^{1/2}}{\hbar^{3}}
$$

\subsection{Magnetic Spin Susceptibility - Pauli Paramagnetism}
The Hamiltonian including the Zeeman interaction term is given by
$$
H = \frac{\textbf{p}^{2}}{2m} + g\mu_{B}\textbf{B}\cdot\sigma
$$
The energies of the two spin configurations are therefore
$$
\epsilon(\textbf{k}, \uparrow) = \frac{\hbar^{2}|\textbf{k}|^{2}}{2m} + \mu_{B}B
$$
$$
\epsilon(\textbf{k}, \downarrow) = \frac{\hbar^{2}|\textbf{k}|^{2}}{2m} - \mu_{B}B
$$
The expectation value of the energy for a single spin is given by
$$E = \frac{1}{Z}\left [ \epsilon(\uparrow)e^{-\beta \epsilon(\uparrow)} + \epsilon(\downarrow)e^{-\beta \epsilon(\downarrow)}\right ]$$

or we could make use of the free energy definition
$$F = \frac{-1}{\beta}\ln Z$$

Either way, the spin magnetization (the moment per unit volume) is given by

$$
M = -\frac{1}{V}\frac{\partial E}{\partial B} = -\frac{1}{V}\frac{\partial}{\partial B}\left [ -\frac{1}{\beta}\ln Z\right ]
$$
After a few steps, we obtain
$$
M = \frac{N\mu_{B}}{VZ}\left [  -e^{-\mu_{B}B\beta} + e^{\mu_{B}B\beta}\right ] = \frac{\mu_{B}}{V}[(\# \downarrow) - (\# \uparrow)]
$$

It is energetically favorable for the spins to be pointing down, so the higher number of down spins results in a magnetization developing in the same direction
as the applied field. This process is known as \emph{Pauli paramagnetism}, where the Pauli name indicates the spin magnetization of the
free electron gas.

When $B = 0$, the density of states $g_{\uparrow}(\epsilon)$ and $g_{\downarrow}(\epsilon)$ are the same. That is
$$
g_{\uparrow}(\epsilon) = g_{\downarrow}(\epsilon) = g(\epsilon)/2 \propto \epsilon^{1/2}
$$

When $B \neq 0$, the densities of states $g_{\uparrow}(\epsilon)$ and $g_{\downarrow}(\epsilon)$ are shifted to the right and left
respectively by $\mu_{B}B$. The $\uparrow$ states are pushed above the Fermi level, but they may lower their energy by flipping. The number of spins that
flip is given roughly by $g_{\uparrow}(E_{F})\mu_{B}B$. This argument is applicable because the number of electrons in the system does not change (and neither does
the temperature), and therefore neither does the chemical potential $\mu = E_{F}$.

Given this approximation for the number of flipped spins, the magnetization is therefore given by

$$
M = \mu_{B}[2g_{\uparrow}(E_{F})\mu_{B}B] = g(E_{F})\mu_{B}^{2}B
$$

Furthermore, the magnetic susceptibility is given by
$$
\chi = \lim_{H \rightarrow 0}\frac{\partial M}{\partial H}
$$
where $H = B/\mu_{0}$. So,
$$
\chi_{Pauli} = \frac{dM}{dH} = \mu_{0}\frac{dM}{dB} = \mu_{0}\mu_{B}^{2}g(E_{F})
$$


\subsubsection{Sommerfeld failures}
\begin{itemize}
  \item The mean free path is quite long - $l \approx 10^{2}$ angstroms. How do electrons avoid scattering from the ions in-between?
  \item To what degree do core electrons not "count" in calculating Fermi energies and velocities? What about insulators which have no free
  electrons?
  \item Why does the Hall coefficient change sign based on charge carrier?
  \item The features of the optical spectra of metals are not explained at all.
  \item The measure specific heat of metals is still off in some cases (and, to the same degree, so are measurements of
  mass of electrons in these metals).
  \item Why are some metals magnetic without an applied external field?
  \item What about the effect of the ignored electron-electron interactions?
\end{itemize}

\section{Quantum Hall Effect and Landau Quantization}
See: http://www.damtp.cam.ac.uk/user/tong/qhe/qhe.pdf
\subsection{Landau Levels}
In this section, we will review the behavior of free particles moving in a magnetic field and the resulting phenomenon of
Landau levels. First a note about spin: we can neglect spin consideration mostly because in the large field limit, the energy required to flip a
spin becomes very large. So, for low energy systems, the electrons behave as if they are spinless.

The quantum Hamiltonian in the presence of a magnetic field $\textbf{B} = \nabla \times \textbf{A}$ is
$$
H = \frac{1}{2m}(\textbf{p} + e\textbf{A})^{2}
$$
where the mechanical momentum is $\textbf{p} = -i\hbar\nabla$.

Note: this result may be obtained by writing down the classical Lagrangian

$$ L = \frac{1}{2}m\dot{\textbf{x}}^{2} - (-e(\dot{\textbf{x}} \times \textbf{B})) = \frac{1}{2}m\dot{\textbf{x}}^{2} + e(\dot{\textbf{x}} \cdot \textbf{A})$$ ?? Why?
The canonical momentum is therefore
$$
\frac{\partial L}{\partial \dot{\textbf{x}}} = m\dot{\textbf{x}} - e\textbf{A}
$$
and the Hamiltonian is given by
$$
H = \dot{\textbf{x}}\frac{\partial L}{\partial \dot{\textbf{x}}} - L
$$

If we let $\textbf{P} = \textbf{p} + e\textbf{A}$, then
$$
H = \frac{1}{2m}(\textbf{P})^{2}
$$
and $[P_{i}, P_{j}] = -ie\hbar B$ for $i \neq j$. Now, if we define
$$
a, a^{\dagger} = \frac{1}{\sqrt{2e\hbar B}}(P_{x} \pm i P_{y})
$$
Then,
$$
H = \hbar \frac{eB}{m}\left (a^{\dagger}a + \frac{1}{2} \right) = \hbar \omega_{B}\left(a^{\dagger}a + \frac{1}{2}\right )
$$
where $\omega_{B}$ is the cyclotron frequency. The well-known result is that the eigenstates $\ket{n}$ have
energies
$$E_{n} = \hbar \omega_{b} \left(n + \frac{1}{2}\right)$$
where each energy level is known as a \emph{Landau level}.

\subsection{The Landau Gauge}
See Landau levels problem, homework 2.

In the Landau gauge, $\textbf{A} = Bx\hat{y}$.

The Hamiltonian then becomes
$$
H = \frac{1}{2m}(\textbf{p} + exB\hat{y})^{2} = \frac{1}{2m}\left ( p_{x}^{2} + (p_{y} + exB)^{2}\right )
$$
We can note that, since $p_{y}$ and $x$ commute, this Hamiltonian can be broken up into a form $H_{x} + H_{y}$ and
therefore, it must admit solutions of the form
$$\psi(x,y) = \psi_{x}(x)\psi_{y}(y)$$
Since $H_{y}$ is simply the Hamiltonian of a free particle, we can see that the solutions should at least be
$$
\psi(x,y) = e^{ik_{y}y}\psi_{x}(x)
$$
Acting on this solution with the Hamiltonian will replace $p_{y}$ with $\hbar k_{y}$ so we obtain
$$
H = \frac{1}{2m}\left ( p_{x}^{2} + (\hbar k_{y} + exB)^{2}\right ) = \frac{p_{x}^{2}}{2m} + \frac{1}{2m}(eB)^{2}(\frac{\hbar k_{y}}{eB} + x)^{2}  )
$$

This is simply the Hamiltonian of a harmonic oscillator with frequency $\omega = (eB/m)$ where the zero of potential has been shifted to $x = -\hbar k_{y}/eB$. Shifting the zero of
potential does not disturb the energy spectrum. Therefore, the eigenenergies are given by
$$
E_{\nu} = \hbar \omega(\nu + \frac{1}{2})
$$
and that means that $\psi_{x}(x)$ must be the eigenfunctions satisfying the 1D harmonic oscillator.

An important observation is that the wavefunctions depend on two numbers: $k_{y} and \nu$, but the eigenvalues depend only on $\nu$.

Let us impose the volume restriction of $L_{x}$ and $L_{y}$. The restriction on the $\hat{y}$-direction indicates that $k_{y}$ is
quantized such that
$$
k_{y} = \frac{2\pi n}{L}
$$
The restriction on $\hat{x}$ indicates that the center of the potential cannot fall outside of this range. Therefore,

$$
0 \leq \hbar k_{y}{eB} \leq L_{x}
$$

Therefore,
$$
0 \leq k_{y} \leq L_{x}eB/\hbar
$$

And the total number of states is given by
$$
\mathcal{N} = \frac{L_{y}}{2\pi} \int_{0}^{L_{x}eB/\hbar} dk_{y} =  \frac{L_{y}L_{x}eB}{2\pi\hbar}
$$

This is the degeneracy of each Landau level. Now, let

$$
\mathcal{N} = \frac{L_{x}L_{y}B}{\Phi_{0}}
$$

where the \emph{quantum of flux} $\Phi_{0} = \frac{2\pi\hbar}{e}$
\subsection{Quantum Hall Effect}

At low temperatures and strong magnetic fields, quantum hall effects dominate. There are two flavors:
\begin{itemize}
  \item Integer QHE (von Klitzing paper):
  \item Fractional QHE
\end{itemize}

\subsubsection{Integer Quantum Hall Effect}
The integer quantum Hall effect is characterized by the following features:
\begin{itemize}
  \item The Hall resistivity $\rho_{xy}$ is nearly quantized with values
  $$
    \rho_{xy} = \frac{2\pi\hbar}{e^{2}}\frac{1}{\nu}
  $$
  where $\nu$ is an integer.
  \item Comparing to the classical result, Drude predicts
  $$
    \rho_{xy} = \frac{B}{ne}
  $$
  To obtain the quantized values of $\rho$, we required the charge carrier density to be
  $$
    n = \frac{B}{e}\nu\frac{e^{2}}{2\pi\hbar} = \frac{eB\nu}{2\pi\hbar} = \frac{B\nu}{\Phi_{0}}
  $$
  This is the precise number of electrons (per unit area) needed to fill the first $\nu$ Landau levels.
  \item So the Hall resistivity is inversely proportional to the number of filled Landau levels, and both proportional and inversely proportional to $B$. Therefore, there is no $B$-dependence. Only dependence on $\nu$
\end{itemize}

\section{Elastic Waves in Crystals}
We will consider the properties of a crystal when viewed as a continuuous medium. This is
a valid view for wavelengths longer than $10^{-6} cm$ or frequencies below $10^{12} Hz$.

The basic concepts centers around Hooke's law: for small strains, then the strain in elastic
solids is directly proportional to the stress.

A uniform deformation in a solid may be defined by the coefficients $e_{\alpha, \beta}$ where
the new axes $\textbf{x}', \textbf{y}', \textbf{z}'$ are given by
$$\textbf{x}' = (1+\epsilon_{xx})\hat{x} + \epsilon_{xy}\hat{y} + \epsilon_{xz}\hat{z}$$
$$\textbf{y}' = \epsilon_{yx}\hat{x} + (1+\epsilon_{yy})\hat{y} + \epsilon_{yz}\hat{z}$$
$$\textbf{z}' = \epsilon_{zx}\hat{x} + \epsilon_{zy}\hat{y} + (1+\epsilon_{zz})\hat{z}$$

The displacement $\textbf{R}$ of the deformation is defined by
$$\textbf{R} = \textbf{r}' - \textbf{r} = x(\textbf{x}' - \hat{x}) + y(\textbf{y}' - \hat{y}) +z(\textbf{z}' - \hat{z})$$
More explicitly,
\begin{equation}
\begin{aligned}
  \textbf{R} = & (x\epsilon_{xx} + y\epsilon_{yx} + z\epsilon_{zx})\hat{x} \\
               & + (x\epsilon_{xy} + y\epsilon_{yy} + z\epsilon_{zy})\hat{y} \\
               & + (x\epsilon_{xz} + y\epsilon_{yz} + z\epsilon_{zz})\hat{z}
\end{aligned}
\end{equation}
Which can be written as

$$\textbf{R}(\textbf{r}) = u(\textbf{r})\hat{x} + v(\textbf{r})\hat{y} + w(\textbf{r})\hat{z}$$

It is usual to work with the following components to define the \emph{strain}:
$$e_{1} = e_{xx} \equiv \epsilon_{xx} = \frac{\partial u}{\partial x}$$
$$e_{2} = e_{yy} \equiv \epsilon_{yy} = \frac{\partial v}{\partial y}$$
$$e_{3} = e_{zz} \equiv \epsilon_{zz} = \frac{\partial w}{\partial z}$$
$$e_{4} = e_{xy} \equiv \epsilon_{yx} + \epsilon_{xy} = \frac{\partial u}{\partial y} + \frac{\partial v}{\partial x}$$
$$e_{5} = e_{yz} \equiv \epsilon_{zy} + \epsilon_{yz} = \frac{\partial v}{\partial z} + \frac{\partial w}{\partial y}$$
$$e_{6} = e_{zx} \equiv \epsilon_{zx} + \epsilon_{xz} = \frac{\partial u}{\partial z} + \frac{\partial w}{\partial x}$$
where the equalities only hold for small strains (neglecting $\mathcal{O}(\epsilon^{2})$).

The \emph{dilation} is the fractional increase of volume associated with a deformation and is given by
$$
\delta \equiv e_{xx} + e_{yy} + e_{zz}
$$

The force acting on a unit area is defined as the \emph{stress}. The nine stress components are given by
$\sigma_{ij}$, which represents a force acting in the $\hat{i}$ direction on a face whose normal is in the $\hat{j}$
direction. Note: if a body is in static equilibrium (or simply not rotating), the total torque about the origin must vanish.
Therefore, $\sigma_{ij} = \sigma_{ji}$ and there are only six, instead of nine, independent stress components.

\begin{itemize}
  \item The dimensions of stress are force per unit area, or energy per unit volume.
  \item Strain is dimensionless.
\end{itemize}

Hooke's law states that for small deformations, the strain is directly proportional to the stress.
\begin{itemize}
  \item The \emph{elastic compliance constants} \textbf{S} (a $6\times6$ matrix) satisfy
  $\textbf{e} = \textbf{S}$\textbf{$\sigma$}
  \item The \emph{elastic stiffness constants} \textbf{C}  (a $6\times6$ matrix) satisfy
  \textbf{$\sigma$} $ = \textbf{C}\textbf{e}$
  \item The elastic stiffness constants are symmetrical: $C_{\alpha\beta} = C_{\beta\alpha}$
\end{itemize}

The \emph{elastic energy density } $U$ is given by
$$
U = \frac{1}{2}\sum_{\lambda = 1}^{6}\sum_{\mu = 1}^{6}\tilde{C}_{\lambda\mu}e_{\lambda}e_{\mu}
$$
where $\tilde{C} = C$? The stress components $\sigma_{ij}$ are found by taking the derivative of $U$ with the
associated strain component. For example,

$$\sigma_{xx} = \frac{\partial U}{\partial e_{xx}} = \frac{\partial U}{\partial e_{1}}$$

For a cubic crystal, symmetry allows
$$
\begin{bmatrix}
C_{11} & C_{12}  & C_{12} & 0  & 0 & 0\\
C_{12} & C_{11} & C_{12} & 0 & 0 & 0\\
C_{12} & C_{12} & C_{11} & 0 &  0& 0\\
0 & 0 & 0 & C_{44} & 0 & 0\\
0 &  0&  0&  0&  C_{44}& 0\\
0 &0  &  0& 0 &0  & C_{44}
\end{bmatrix}
$$
\begin{itemize}
  \item The \emph{bulk modulus} $B$ is defined by the relation
  $$U = \frac{1}{2}B\delta^{2}$$
  for a uniform dilation $e_{xx} = e_{yy} = e_{zz} = \frac{1}{3}\delta$.
  \item The \emph{compressibility } K is defined by
  $$K = \frac{1}{B}$$
\end{itemize}

\subsection{Elastic Waves}
The net x-component of the force on a cubic crystal of volume $\Delta x \Delta y \Delta z$, and mass $\rho \Delta x \Delta y \Delta z $ is given by

$$
F_{x} = \left ( \frac{\partial \sigma_{xx}}{\partial x} + \frac{\partial \sigma_{xy}}{\partial y} + \frac{\partial \sigma_{xz}}{\partial z} \right)\Delta x \Delta y \Delta z
$$

The equation of motion in the x-direction is therefore
$$\rho \ddot{u} = \left ( \frac{\partial \sigma_{xx}}{\partial x} + \frac{\partial \sigma_{xy}}{\partial y} + \frac{\partial \sigma_{xz}}{\partial z} \right)$$
and similar equations follow for $v$ and $w$.

\subsection{Waves in [100]}
\subsubsection{Longitudinal}
Consider a longitudinal wave given by
$$ u = u_{0}e^{i(kx - \omega t)}$$

where the wavevector and the particle motion are both along the $x$ cube edge. The wavevector is given by $k = \frac{2\pi}{\lambda}$ and $\omega = 2\pi\nu$ is the
angular frequency. The velocity of the longitudinal wave is given by $\nu \lambda = \omega/k$. This value can be found by inserting the form of the longitudinal
wave into the equation of motion.

\subsubsection{Transverse}
Consider a transverse wave where the wavevector is direction along $\hat{x}$ and the particle motion is along $\hat{y}$. The wave is then given by
$$v = v_{0}e^{i(kx - \omega t)}$$

The velocity can likewise be found as $\nu \lambda = \omega/k$. Note: for $\textbf{k}$ parallel to [100], the particle velocity will be the same regardless of
whether particle motion is in $\hat{y}$ or $\hat{z}$. This is not necessarily the case for arbitrary $\textbf{k}$.

\subsection{Waves in [110] }
Consider a shear wave propogating in the $xy$-plane with particle motion along $\hat{z}$:
$$
w = w_{0}e^{i(k_{x}x + k_{y}y - \omega t)}
$$

For a shear wave propagating in the $xy$-plane with particle motion in the $xy$-plane, there are two equations of interest:
$$
u = u_{0}e^{i(k_{x}x + k_{y}y - \omega t)}
$$
$$
v = v_{0}e^{i(k_{x}x + k_{y}y - \omega t)}
$$
Inserting these two equations into the corresponding equations of motion gives two coupled equations in $u$ and $v$. Making use of the
face that $k_{x} = k_{y} = k/\sqrt{2}$, and requiring that the determinant of the coefficients of $u$ and $v$ in the coupled equations vanish, two roots
for $(\omega/k)^{2}$ can be found: one describing a longitudinal wave (along [110]), the other describing a transverse wave (along [1-10]).

In the special propagation directions of [100], [110], and [111], there are three normal modes with the following polarizations: one longitudinal and two transverse.
For [100] and [111], the two transverse velocities will equal one another. Sound waves are longitudinal waves.
\end{document}
