\title{Solid State Physics: Midterm Review}

\documentclass[10pt]{article}
\usepackage{amsthm}
\usepackage{graphicx}
\usepackage{subfig}
\usepackage{physics}
\graphicspath{ {figures/} }
\begin{document}
\maketitle

\section{Drude Theory}
\subsection{Basic Model}
\begin{itemize}
  \item Theory of metallic conduction proposed by Paul Drude.
  \item Considers metal a "gas" of electrons on which the kinetic theory of gases may be applied.
  \item Assumption of kinetic theory of gases: molecules, or electrons, are identical constituents that move
  in straight lines until they encounter a collision.
  \item Assumption of metallic "gases": when atoms of a metallic element form a metal, the valence (or "conduction")
  electrons move freely while the positively charged ions remain stationary.
  \item Reality: metals have a much higher density than gas and include very strong electron-electron and electron-ion
  interactions.

\end{itemize}

The \emph{charge carrier density} $n$ may be written as
$$
n = \frac{N}{V} = N_{A}Z\frac{\rho}{A}
$$
where $N$ is the total number of conduction electrons, $V$ is the sample volume, $N_{A}$ is Avogradro's number,
$Z$ is the number of valence electrons per atom, $\rho$ is the mass density of the metal, and $A$ is the atomic mass (mass per mole of
the metal). A typical charge carrier density is on the order of $n ~ 10^{22} cm^{-3}$ for metals.

Another useful way to characterize the charge carrier density is by specifying $r_{s}$, the radius of the spherical volume containing
one charge carrier.
$$
\frac{V}{N} = \frac{1}{n} = \frac{4}{3}\pi r_{s}^{3} \rightarrow r_{s} = (\frac{3}{4\pi n})^{1/3}
$$

The basic assumption of the Drude theory are:
\begin{itemize}
  \item \emph{Indepedent electron approximation}: electrons do not interact with one another (between collisions).
  \item \emph{Free electron approximation}: electrons do not interact with positively charged ions (between collisions).
  \item Collisions are instantaneous events that abruptly alter the electron velocity. The mechanism of scattering is not
  well-defined. Drude assumed it was only due to electron-ion scattering, but the reality is much more complicated.
  \item Electrons experience a collision with probability $1/\tau$ per unit time. The probability of a collision in time $dt$
  is $dt/\tau$. $\tau$ is the \emph{mean free time}, or \emph{relaxation time}.
  \item Electrons achieve thermal equilibrium only through collision. After each collision, an electron emerges with a randomly oriented
  velocity proportional to the local temperature $T$.
\end{itemize}

\subsection{DC Conductivity}
Ohm's Law:
$$V = IR$$
where $V$ is the voltage drop, $I$ is the current, and $R$ is the resistance. Using $\rho = RA/L$, where $L$ is the
distance over which the voltage drop occurs, and $A$ is the cross-sectional area of the sample,
$$\textbf{E} = \rho \textbf{j} $$
where $\textbf{j}$ is the current density - the amount of charge flowing through a unit area per unit time.
For $n$ electrons moving with velocity $\textbf{v}$,
$$
\textbf{j} = nq\textbf{v} = n(-e)\textbf{v}
$$
\begin{itemize}
  \item When $\textbf{E} = 0$, there is no preferred direction so $\langle \textbf{v} \rangle = 0$ and therefore $\textbf{j} = 0$.
  \item When $\textbf{E} \neq 0$, the velocity of an electron is given by
  $$ \textbf{v} = \textbf{v}_{0} + \frac{(-e)\textbf{E}t}{m}$$
  So,
  $$ \langle \textbf{v} \rangle = \langle \textbf{v}_{0} \rangle + \frac{(-e)\textbf{E}\tau}{m} = \frac{(-e)\textbf{E}\tau}{m} $$
  Where the velocity acquired over time $t$ due to a constant acceleration of $\textbf{a} = \frac{-e\textbf{E}}{m}$ is given by $\textbf{at}$.
\end{itemize}
We can therefore generally write the current density as
$$\textbf{j} = \frac{ne^{2}\tau}{m} \textbf{E}= \sigma \textbf{E} $$
where the \emph{conductivity} $\sigma$ is given by $ne^{2}\tau/m$.
If we consider the \emph{mean free path}, $l = \langle v\rangle \tau$, Drude's predictions for $\langle v\rangle$ based on kinetic theory
of gases estimate $l ~ 5 $angstroms. This is consistent with Drude's view that electron-ion collisions dominated. In reality, the mean free
path is at least an order of magnitude larger. At low temperatures, the mean free path is underestimated by a factor of $~10^{3}$.

\subsection{Drude Equation of Motion}
The time-dependent current density may be written in terms of the momentum $\textbf{p}(t)$ as
$$\textbf{j}(t) = n(-e)\textbf{v}(t) = -\frac{ne}{m}\textbf{p}(t)$$
At an infinitesimal time $dt$ later, the momentum will be given by
$$\textbf{p}(t + dt) = \left ( 1 - \frac{dt}{\tau}\right ) \left [ \textbf{p}(t) + \textbf{F}dt\right ] + \left ( \frac{dt}{\tau}\right )\mathcal{O}(\textbf{F}dt)$$
The first term comes from the fact that, with probability $1-dt/\tau$, the electron will not collide during time $dt$ and will therefore acquire the impulse $\textbf{F}dt$
in addition to its original momentum, $\textbf{p}(t)$. The second term comes from the case that the electron does collide, which happens with probability $dt/\tau$, in
which case its contribution to the momentum is $\mathcal{O}(\textbf{F}dt)$ since its velocity after collision averages over all directions to zero.
Therefore,
$$\textbf{p}(t + dt) = \textbf{p}(t) - \frac{\textbf{p}(t)dt}{\tau} + \textbf{F}dt + \mathcal{O}(dt^{2})$$
Ignoring the higher-order terms in $dt$, we obtain:
$$\frac{\textbf{p}(t + dt) - \textbf{p}(t)}{dt} = \frac{d\textbf{p}(t)}{dt} =  - \frac{\textbf{p}(t)}{\tau} + \textbf{F}$$

\subsection{Hall Effect and Magnetoresistance}
E.H. Hall predicted that the force experienced by a current due to a magnetic field would manifest as an observable transverse voltage, today known
as the Hall voltage.
The experimental setup is an applied electric field $E_{x}$ driving current in the $+\hat{x}$ direction along a sample. An applied magnetic field
is also present, $\textbf{B} = B\hat{z}$ perpendicular to the sample. Charge carriers experience a Lorentz force given by
$$qE_{x}\hat{x} + q(\textbf{v}\times\textbf{B})$$
In the case of electrons, $q = -e$, and the second term gives rise to a deflection in the $-\hat{y}$ direction. As electrons build up on the sample
surface along the $-\hat{y}$ edge, a transverse electric field $E_{y}$ is created in the $-\hat{y}$ direction. This field will grow in magnitude until is opposes
the magnetic force term. Therefore,
$$qE_{y}\hat{y} + q(\textbf{v}\times\textbf{B}) = 0$$
In the case of electrons,
$$-eE_{y} + -ev_{x}B = 0$$
Therefore, the transverse electric field has magnitude $E_{y} = -v_{x}B$.

There are two quantities of interest:
\begin{itemize}
  \item \emph{Magnetoresistance} $\rho(B) = \frac{E_{x}}{j_{x}}$. This quantity is field independent.
  \item \emph{Hall coefficient} $R_{H} = \frac{E_{y}}{j_{x}B}$. This value is negative for negative charge carriers
  and positive for positive charge carriers. Measurement of the Hall coefficient, or Hall field $E_{y}$, gives information
  about the charge carriers.
\end{itemize}

Starting with the Drude equation of motion, we can derive equations for the magnetoresistance and Hall coefficient.
$$\frac{d\textbf{p}(t)}{dt} =  - \frac{\textbf{p}(t)}{\tau} + \textbf{F} = \frac{\textbf{p}(t)}{\tau} + (-e\textbf{E} + \frac{-e}{m})(\textbf{p}\times\textbf{B})$$
In the steady state, $\frac{d\textbf{p}}{dt} = 0$, so we obtain
$$
0 = -\frac{p_{x}}{\tau} + (-eE_{x} + \frac{-e}{m}p_{y}B) = -eE_{x} - \omega p_{y} - \frac{p_{x}}{\tau}
$$
$$
0 = -\frac{p_{y}}{\tau} + (-eE_{y} - \frac{-e}{m}p_{x}B) = -eE_{y} + \omega p_{x} - \frac{p_{y}}{\tau}
$$
where $\omega = eB/m$, the cyclotron frequency. The quantity $\omega \tau$ is a good
measure of the strength of the magnetic field. For example, if $\omega \tau$ is very small, the current
is very nearly parallel to $E_{x}$.  From these we obtain
$$
\sigma_{0}E_{x} = \omega j_{y}\tau + j_{x}
$$
$$
\sigma_{0}E_{y} = - \omega j_{x} \tau + j_{y}
$$
The Hall voltage is obtained in the limit when $j_{y} = 0$. Therefore, the steady state transverse field is given by
$$E_{y} = \frac{- \omega j_{x}\tau}{\sigma_{0}} = -\frac{B}{ne}j_{x}$$
So the Hall coefficient is given by
$$R_{H} = \frac{E_{y}}{j_{x}B} = -\frac{1}{ne}$$
This result predicts that the Hall coefficient is dependent only on the number of charge carriers. However, the actual value is
temperature and field-dependent. But the limiting value, in the case of very low temperature and very high fields, is the Drude
result.

\subsection{AC Electrical Conductivity}
We wish to calculate the current induced by a time-dependent electric field. The field can be written in the form
$$
\textbf{E}(t) = Re(\textbf{E}(\omega)e^{-i\omega t})
$$
In the absence of a magnetic field, the equation of motion becomes
$$
\frac{d\textbf{p}(t)}{dt} =  - \frac{\textbf{p}(t)}{\tau} + \textbf{F} = \frac{\textbf{p}(t)}{\tau} + -e\textbf{E}
$$
And the solution we seek is therefore of the form
$$
\textbf{p}(t) = Re(\textbf{p}(\omega)e^{-i\omega t})
$$
The equation of motion gives us
$$
-i\omega \textbf{p}(\omega) = -\frac{\textbf{p}(\omega)}{\tau} - e\textbf{E}(\omega)
$$
And we can use the fact that $\textbf{j} = -ne\textbf{p}/m$ to write
$$
\textbf{j}(t) = Re(\textbf{j}(\omega)e^{-i\omega t})
$$
and then rewrite the equation of motion as
$$
i\frac{m\omega}{ne}\textbf{j}(\omega) = \frac{m\textbf{j}(\omega)}{ne\tau} - e\textbf{E}(\omega)
$$
$$
\left[ i\frac{m\omega}{ne} - \frac{m}{ne\tau} \right ]\textbf{j}(\omega) = - e\textbf{E}(\omega)
$$
$$
\textbf{j}(\omega) = \frac{-ne^{2}}{m(i\omega - \frac{1}{\tau})}\textbf{E}(\omega)
$$
We finally obtain the expression for the frequency-dependent AC conductivity $\sigma(\omega)$:
$$
\textbf{j}(\omega) = \sigma(\omega)\textbf{E}(\omega)
$$
where
$$
\sigma(\omega) = \frac{\sigma_{0}}{1-i\omega\tau}
$$
In the case of $\omega = 0$, this reduces to the DC conductivity.
There are two complications addressed in Ashcroft/Mermin:
\begin{itemize}
  \item Neglecting the $\textbf{B}$ counterpart: this is acceptable because the term is technically scaled
  by a factor of $v/c$. In most cases, it is roughly $10^{-10}$ of the $\textbf{E}$-field term and may safely be ignored.
  \item We have assumed spatial uniformity, but the $\textbf{E}$-field actually varies in time and space. However, as long as
  the field does not vary much over a few mean free path lengths, we may safely assume spatial invariance.
\end{itemize}

\subsection{Thermal Conductivity}
The Wiedemann-Franz law states that the ratio of the thermal to the electrical conductance, $\kappa / \sigma$, is
directly proportional to the temperature.

The Drude model assumes the bulk of thermal conductance is performed by conduction electrons.

\section{Sommerfeld Theory}
Sommerfeld combined Fermi-Dirac statistics with the Drude model.
\subsection{Basic Fermi-Dirac Statistics}
Given a system of free electrons with chemical potential $\mu$, the probability of an eigenstate of energy $E$ bring occupied is given by the
Fermi factor
$$n_{F}(\beta(E-\mu)) = \frac{1}{e^{\beta(E - \mu)}+1}$$
At very low temperature, $n_{F} \approx 1$ for $E < \mu$ and $n_{F} \approx 0$ for $E > \mu$. So, $n_{F}$ is a step function
at low $T$. At higher $T$, it becomes more spread out.

For electrons in a box of size $V = L^{3}$, the wavefunctions of the free electron particles are given by
$$\Psi = e^{i\textbf{k}\cdot\textbf{r}}$$
Imposing PBC, we find that $\textbf{k} = (2\pi/L)(n_{1},n_{2},n_{3})$ where $n_{i}$ are integers. The corresponding
energies of the plane waves are
$$\epsilon(\textbf{k}) = \frac{\hbar |\textbf{k}|^{2}}{2m}$$
The total number of electrons in the system, $N$, is given by the sum of all possible states times the probability of each state being occupied.
Therefore,

$$N = 2 \sum_{\textbf{k}}n_{F}(\beta(\epsilon(\textbf{k}) - \mu))$$

where the factor of 2 comes from each state allowing two possible spin configurations. In the continuous limit, we can make
use of the following:

$$\int f(x)dx = \sum f(x)\delta x$$

Therefore, $N$ can be written

$$
N = 2 \frac{L^{3}}{(2\pi)^{3}}\int d\textbf{k} n_{F}(\beta(\epsilon(\textbf{k}) - \mu))
$$

Some definitions:
\begin{itemize}
  \item The \emph{Fermi energy}, $E_{F}$, is the chemical potential at $T = 0$. Sometimes also called the \emph{Fermi Level}.
  In the case of a continuum of states, the Fermi energy is the energy of the most energetic occupied electron state. In the case of a discrete
  set of states, the Fermi energy is the energy halfway between the highest occupied and lowest unoccupied states.
  \item The \emph{Fermi sea} is the set of states filled at $T = 0$.
  \item The \emph{Fermi temperature}, $T_{F}$, is given by $T_{F} = E_{F}/k_{B}$.
  \item The \emph{Fermi wavevector}, $k_{F}$, satisfies
  $$E_{F} = \frac{\hbar^{2}k_{F}^{2}}{2m}$$
  \item The \emph{Fermi momentum} $p_{F} = \hbar k_{F}$.
  \item The \emph{Fermi velocity} $v_{F} = p_{F}/m = \hbar k_{F}/m$.
\end{itemize}

\subsection{Fermi Energy at T = 0}
Consider a 3-dimensional metal with $N$ electrons at $T = 0$. The total number of electrons can be written as

$$
N = 2 \frac{L^{3}}{(2\pi)^{3}}\int d\textbf{k} n_{F}(\beta(\epsilon(\textbf{k}) - \mu))
$$

At $T = 0$, the Fermi function $n_{F}$ becomes a step function that switches at $E_{F}$ such that

$$n_{F} = \Theta(E_{F} - \epsilon(\textbf{k}))$$

Therefore,

$$
N = 2 \frac{L^{3}}{(2\pi)^{3}}\int d\textbf{k} \Theta(E_{F} - \epsilon(\textbf{k})) = 2 \frac{L^{3}}{(2\pi)^{3}} \int^{|k|<k_{F}}d\textbf{k}
$$

The final integral simply yields the volume of a ball of radius $k_{F}$. Therefore,

$$N = 2 \frac{L^{3}}{(2\pi)^{3}} \left (\frac{4}{3}\pi k_{F}^3\right )$$

The surface of this ball is known as the \emph{Fermi surface} - it is the surface dividing filled from unfilled states at $T = 0$.

This result gives us equations for the Fermi wavevector and Fermi energy:
$$$$

\end{document}
