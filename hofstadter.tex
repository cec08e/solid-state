\title{Energy levels and wave functions of Bloch electrons in rational and irrational magnetic fields}

\documentclass[10pt]{article}
\usepackage{amsthm}
\usepackage{graphicx}
\usepackage{subfig}
\usepackage{physics}
\graphicspath{ {figures/} }
\begin{document}
\maketitle

Hofstadter begins with the following question: \emph{how does the rationality or irrationality of
some parameter in a model affect the experimental outcome?}

At first guess, irrationality should seem to have no effect on experimental observables. If this were the case, then the arbitrarily small change in
parameter necessary to create rationality would create a discontinuous physical effect. Said more simply, this effect would require infinite experimental
resolution, which is unreasonable.

The goal, then, is to show that there is a theory in which the rationality or irrationality of the parameter can, in either case, yield continuous physical
observables.

\textbf{The model}: the model in question in a two-dimensional square lattice of spacing $a$, subject to a uniform external magnetic field $H$ perpendicular
to the lattice plane.
\begin{itemize}
  \item We restrict discussion to a single band.
  \item The Bloch energy function is postulated as
  $$W(\vec{k}) = 2E_{0}(\cos(k_{x}a) + \cos(k_{y}a))$$
  \item Peierl's substitution: $\hbar k \ rightarrow \vec{p} - e\vec{A}/c$. Therefore $W$ becomes an operator,
  which is treated as a single-band Hamiltonian. 
\end{itemize}

\end{document}
