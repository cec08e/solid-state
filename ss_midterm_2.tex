\title{Solid State Physics: Midterm 2 Review}

\documentclass[10pt]{article}
\usepackage{amsthm}
\usepackage{graphicx}
\usepackage{subfig}
\usepackage{physics}
\graphicspath{ {figures/} }
\begin{document}
\maketitle

\section{Crystal Structure}
\subsection{Basics of Crystal Structure}
\begin{itemize}
\item A \emph{lattice} is an infinite set of points defined by integer sums of a set of linearly
independent \emph{primitive lattice vectors}.
\item Alternatively: A \emph{lattice} is an infintie discrete set of vectors where addition of
any two vectors in the set gives a third vector in the set, or a \emph{lattice} is an infinite
discrete set of points where the environment of any given point is equivalent to the environment
of any other point.
\item Lattice points may be described in three dimensions as
$$\textbf{R} = n_{1}\textbf{a}_{1} + n_{2}\textbf{a}_2 + n_{3}\textbf{a}_{3}$$
\item In 2+ dimensions, the choice of primitice lattice vectors is not unique.
\end{itemize}

\begin{itemize}
  \item A \emph{unit cell} is the repeated motif which is the elementary building block of
  any periodic structure. When many identical unit cells are tiled together, they
  completely fill all of space and reconstruct the full structure.
  \item A \emph{primitive unit cell} is a unit cell containing exactly one lattice point.
  \item A \emph{conventional unit cell} is a typically less-minimal unit cell, often
  with orthogonal axes, chosen due to ease of use.
  \item A \emph{Wigner-Seitz cell} is constructed by including all points in space that are
  closer to a given lattice point than any other lattice point. Approach: choose a lattice cell
  and draw lines to all possible neighbors. Perpendicular bisectors of these lines bound the
  Wigner-Seitz cell.
  \item The description of the unit cell with respect to the associated lattice point is
  known as the \emph{basis}.
  \item The positions of atoms can be described by
  $$ \textbf{R} = \textbf{R}_{lattice} + \textbf{R}_{basis} $$
  \item The \emph{coordination number} of a lattice is the number of nearest neighbors to any point of the lattice.
\end{itemize}

\subsection{Three-Dimensional Lattices}
The simplest example is the \emph{simple cubic} lattice.
\begin{itemize}
  \item The primitive unit cell is typically a cube, where each corner contributes $1/8$ lattice point.
  \item \emph{Tetragonal} and \emph{orthorhombic} lattices have two and three different primitive lattice
  vector lengths, respectively.
  \item A point in the lattice may be written as $[uvw]$ where
  $$[uvw] =  u\textbf{a}_{1} + v\textbf{a}_2 + w\textbf{a}_{3}$$
  \item In the case of orthogonal axes, $\textbf{a}_{1}$ is assumed to be in the $\hat{x}$ direction, etc.
\end{itemize}

The \emph{body-centered cubic} lattice is identical to the simple cubic lattice, but with an additional
lattice point in the center of the cube.
\begin{itemize}
  \item The conventional unit cell contains $8(1/8) + 1 = 2$ lattice points.
  \item The points of the BCC lattice can be written as
  $$\textbf{R}_{corner} = [n_{1}, n_{2}, n_{3}]$$
  $$\textbf{R}_{center} = [n_{1}, n_{2}, n_{3}] + [1/2, 1/2, 1/2]$$
  \item We can also think of the BCC as two interpenetrating SC lattices, displaced by $[1/2, 1/2, 1/2]$.
  \item The coordination number $Z = 8$.
  \item One choice of primitive lattice vectors:
  $$\textbf{a}_{1} = [1, 0, 0]$$
  $$\textbf{a}_{2} = [0, 1, 0]$$
  $$\textbf{a}_{3} = [1/2, 1/2, 1/2]$$

\end{itemize}

The \emph{face-centered cubic} lattice is similar to the simple cubic lattice, but with the addition of a
lattice point on every cube face.
\begin{itemize}
\item The conventional unit cell contains $8(1/8) + 6(1/2) = 4$ lattice points.
\item The points of the FCC lattice can be written as
$$\textbf{R}_{corner} = [n_{1}, n_{2}, n_{3}]$$
$$\textbf{R}_{xy} = [n_{1}, n_{2}, n_{3}] + [1/2, 1/2, 0]$$
$$\textbf{R}_{yz} = [n_{1}, n_{2}, n_{3}] + [0, 1/2, 1/2]$$
$$\textbf{R}_{zx} = [n_{1}, n_{2}, n_{3}] + [1/2, 0, 1/2]$$
\item We can also think of the FCC as four interpenetrating SC lattices, displaced as described above.
\item One choice of primitive lattice vectors:
$$\textbf{a}_{1} = [1/2, 1/2, 0]$$
$$\textbf{a}_{2} = [1/2, 0, 1/2]$$
$$\textbf{a}_{3} = [0, 1/2, 1/2]$$



\end{itemize}

\section{Reciprocal Lattice, Brillouin Zone, Waves in Crystals}


\end{document}
